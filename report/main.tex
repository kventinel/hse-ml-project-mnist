\documentclass[11pt]{article}

    \usepackage[breakable]{tcolorbox}
    \usepackage{parskip} % Stop auto-indenting (to mimic markdown behaviour)
    
    \usepackage{iftex}
    \ifPDFTeX
    	\usepackage[T1]{fontenc}
    	\usepackage{mathpazo}
    \else
    	\usepackage{fontspec}
    \fi

    % Basic figure setup, for now with no caption control since it's done
    % automatically by Pandoc (which extracts ![](path) syntax from Markdown).
    \usepackage{graphicx}
    % Maintain compatibility with old templates. Remove in nbconvert 6.0
    \let\Oldincludegraphics\includegraphics
    % Ensure that by default, figures have no caption (until we provide a
    % proper Figure object with a Caption API and a way to capture that
    % in the conversion process - todo).
    \usepackage{caption}
    \DeclareCaptionFormat{nocaption}{}
    \captionsetup{format=nocaption,aboveskip=0pt,belowskip=0pt}

    \usepackage[Export]{adjustbox} % Used to constrain images to a maximum size
    \adjustboxset{max size={0.9\linewidth}{0.9\paperheight}}
    \usepackage{float}
    \floatplacement{figure}{H} % forces figures to be placed at the correct location
    \usepackage{xcolor} % Allow colors to be defined
    \usepackage{enumerate} % Needed for markdown enumerations to work
    \usepackage{geometry} % Used to adjust the document margins
    \usepackage{amsmath} % Equations
    \usepackage{amssymb} % Equations
    \usepackage{textcomp} % defines textquotesingle
    % Hack from http://tex.stackexchange.com/a/47451/13684:
    \AtBeginDocument{%
        \def\PYZsq{\textquotesingle}% Upright quotes in Pygmentized code
    }
    \usepackage{upquote} % Upright quotes for verbatim code
    \usepackage{eurosym} % defines \euro
    \usepackage[mathletters]{ucs} % Extended unicode (utf-8) support
    \usepackage{fancyvrb} % verbatim replacement that allows latex
    \usepackage{grffile} % extends the file name processing of package graphics 
                         % to support a larger range
    \makeatletter % fix for grffile with XeLaTeX
    \def\Gread@@xetex#1{%
      \IfFileExists{"\Gin@base".bb}%
      {\Gread@eps{\Gin@base.bb}}%
      {\Gread@@xetex@aux#1}%
    }
    \makeatother

    % The hyperref package gives us a pdf with properly built
    % internal navigation ('pdf bookmarks' for the table of contents,
    % internal cross-reference links, web links for URLs, etc.)
    \usepackage{hyperref}
    % The default LaTeX title has an obnoxious amount of whitespace. By default,
    % titling removes some of it. It also provides customization options.
    \usepackage{titling}
    \usepackage{longtable} % longtable support required by pandoc >1.10
    \usepackage{booktabs}  % table support for pandoc > 1.12.2
    \usepackage[inline]{enumitem} % IRkernel/repr support (it uses the enumerate* environment)
    \usepackage[normalem]{ulem} % ulem is needed to support strikethroughs (\sout)
                                % normalem makes italics be italics, not underlines
    \usepackage{mathrsfs}
    

    
    % Colors for the hyperref package
    \definecolor{urlcolor}{rgb}{0,.145,.698}
    \definecolor{linkcolor}{rgb}{.71,0.21,0.01}
    \definecolor{citecolor}{rgb}{.12,.54,.11}

    % ANSI colors
    \definecolor{ansi-black}{HTML}{3E424D}
    \definecolor{ansi-black-intense}{HTML}{282C36}
    \definecolor{ansi-red}{HTML}{E75C58}
    \definecolor{ansi-red-intense}{HTML}{B22B31}
    \definecolor{ansi-green}{HTML}{00A250}
    \definecolor{ansi-green-intense}{HTML}{007427}
    \definecolor{ansi-yellow}{HTML}{DDB62B}
    \definecolor{ansi-yellow-intense}{HTML}{B27D12}
    \definecolor{ansi-blue}{HTML}{208FFB}
    \definecolor{ansi-blue-intense}{HTML}{0065CA}
    \definecolor{ansi-magenta}{HTML}{D160C4}
    \definecolor{ansi-magenta-intense}{HTML}{A03196}
    \definecolor{ansi-cyan}{HTML}{60C6C8}
    \definecolor{ansi-cyan-intense}{HTML}{258F8F}
    \definecolor{ansi-white}{HTML}{C5C1B4}
    \definecolor{ansi-white-intense}{HTML}{A1A6B2}
    \definecolor{ansi-default-inverse-fg}{HTML}{FFFFFF}
    \definecolor{ansi-default-inverse-bg}{HTML}{000000}

    % commands and environments needed by pandoc snippets
    % extracted from the output of `pandoc -s`
    \providecommand{\tightlist}{%
      \setlength{\itemsep}{0pt}\setlength{\parskip}{0pt}}
    \DefineVerbatimEnvironment{Highlighting}{Verbatim}{commandchars=\\\{\}}
    % Add ',fontsize=\small' for more characters per line
    \newenvironment{Shaded}{}{}
    \newcommand{\KeywordTok}[1]{\textcolor[rgb]{0.00,0.44,0.13}{\textbf{{#1}}}}
    \newcommand{\DataTypeTok}[1]{\textcolor[rgb]{0.56,0.13,0.00}{{#1}}}
    \newcommand{\DecValTok}[1]{\textcolor[rgb]{0.25,0.63,0.44}{{#1}}}
    \newcommand{\BaseNTok}[1]{\textcolor[rgb]{0.25,0.63,0.44}{{#1}}}
    \newcommand{\FloatTok}[1]{\textcolor[rgb]{0.25,0.63,0.44}{{#1}}}
    \newcommand{\CharTok}[1]{\textcolor[rgb]{0.25,0.44,0.63}{{#1}}}
    \newcommand{\StringTok}[1]{\textcolor[rgb]{0.25,0.44,0.63}{{#1}}}
    \newcommand{\CommentTok}[1]{\textcolor[rgb]{0.38,0.63,0.69}{\textit{{#1}}}}
    \newcommand{\OtherTok}[1]{\textcolor[rgb]{0.00,0.44,0.13}{{#1}}}
    \newcommand{\AlertTok}[1]{\textcolor[rgb]{1.00,0.00,0.00}{\textbf{{#1}}}}
    \newcommand{\FunctionTok}[1]{\textcolor[rgb]{0.02,0.16,0.49}{{#1}}}
    \newcommand{\RegionMarkerTok}[1]{{#1}}
    \newcommand{\ErrorTok}[1]{\textcolor[rgb]{1.00,0.00,0.00}{\textbf{{#1}}}}
    \newcommand{\NormalTok}[1]{{#1}}
    
    % Additional commands for more recent versions of Pandoc
    \newcommand{\ConstantTok}[1]{\textcolor[rgb]{0.53,0.00,0.00}{{#1}}}
    \newcommand{\SpecialCharTok}[1]{\textcolor[rgb]{0.25,0.44,0.63}{{#1}}}
    \newcommand{\VerbatimStringTok}[1]{\textcolor[rgb]{0.25,0.44,0.63}{{#1}}}
    \newcommand{\SpecialStringTok}[1]{\textcolor[rgb]{0.73,0.40,0.53}{{#1}}}
    \newcommand{\ImportTok}[1]{{#1}}
    \newcommand{\DocumentationTok}[1]{\textcolor[rgb]{0.73,0.13,0.13}{\textit{{#1}}}}
    \newcommand{\AnnotationTok}[1]{\textcolor[rgb]{0.38,0.63,0.69}{\textbf{\textit{{#1}}}}}
    \newcommand{\CommentVarTok}[1]{\textcolor[rgb]{0.38,0.63,0.69}{\textbf{\textit{{#1}}}}}
    \newcommand{\VariableTok}[1]{\textcolor[rgb]{0.10,0.09,0.49}{{#1}}}
    \newcommand{\ControlFlowTok}[1]{\textcolor[rgb]{0.00,0.44,0.13}{\textbf{{#1}}}}
    \newcommand{\OperatorTok}[1]{\textcolor[rgb]{0.40,0.40,0.40}{{#1}}}
    \newcommand{\BuiltInTok}[1]{{#1}}
    \newcommand{\ExtensionTok}[1]{{#1}}
    \newcommand{\PreprocessorTok}[1]{\textcolor[rgb]{0.74,0.48,0.00}{{#1}}}
    \newcommand{\AttributeTok}[1]{\textcolor[rgb]{0.49,0.56,0.16}{{#1}}}
    \newcommand{\InformationTok}[1]{\textcolor[rgb]{0.38,0.63,0.69}{\textbf{\textit{{#1}}}}}
    \newcommand{\WarningTok}[1]{\textcolor[rgb]{0.38,0.63,0.69}{\textbf{\textit{{#1}}}}}
    
    
    % Define a nice break command that doesn't care if a line doesn't already
    % exist.
    \def\br{\hspace*{\fill} \\* }
    % Math Jax compatibility definitions
    \def\gt{>}
    \def\lt{<}
    \let\Oldtex\TeX
    \let\Oldlatex\LaTeX
    \renewcommand{\TeX}{\textrm{\Oldtex}}
    \renewcommand{\LaTeX}{\textrm{\Oldlatex}}
    % Document parameters
    % Document title
    \title{Home Project}
    
    \author{Aliaksei Rak\\Nerses Bagiyan}
    
    
    
    
    
% Pygments definitions
\makeatletter
\def\PY@reset{\let\PY@it=\relax \let\PY@bf=\relax%
    \let\PY@ul=\relax \let\PY@tc=\relax%
    \let\PY@bc=\relax \let\PY@ff=\relax}
\def\PY@tok#1{\csname PY@tok@#1\endcsname}
\def\PY@toks#1+{\ifx\relax#1\empty\else%
    \PY@tok{#1}\expandafter\PY@toks\fi}
\def\PY@do#1{\PY@bc{\PY@tc{\PY@ul{%
    \PY@it{\PY@bf{\PY@ff{#1}}}}}}}
\def\PY#1#2{\PY@reset\PY@toks#1+\relax+\PY@do{#2}}

\expandafter\def\csname PY@tok@w\endcsname{\def\PY@tc##1{\textcolor[rgb]{0.73,0.73,0.73}{##1}}}
\expandafter\def\csname PY@tok@c\endcsname{\let\PY@it=\textit\def\PY@tc##1{\textcolor[rgb]{0.25,0.50,0.50}{##1}}}
\expandafter\def\csname PY@tok@cp\endcsname{\def\PY@tc##1{\textcolor[rgb]{0.74,0.48,0.00}{##1}}}
\expandafter\def\csname PY@tok@k\endcsname{\let\PY@bf=\textbf\def\PY@tc##1{\textcolor[rgb]{0.00,0.50,0.00}{##1}}}
\expandafter\def\csname PY@tok@kp\endcsname{\def\PY@tc##1{\textcolor[rgb]{0.00,0.50,0.00}{##1}}}
\expandafter\def\csname PY@tok@kt\endcsname{\def\PY@tc##1{\textcolor[rgb]{0.69,0.00,0.25}{##1}}}
\expandafter\def\csname PY@tok@o\endcsname{\def\PY@tc##1{\textcolor[rgb]{0.40,0.40,0.40}{##1}}}
\expandafter\def\csname PY@tok@ow\endcsname{\let\PY@bf=\textbf\def\PY@tc##1{\textcolor[rgb]{0.67,0.13,1.00}{##1}}}
\expandafter\def\csname PY@tok@nb\endcsname{\def\PY@tc##1{\textcolor[rgb]{0.00,0.50,0.00}{##1}}}
\expandafter\def\csname PY@tok@nf\endcsname{\def\PY@tc##1{\textcolor[rgb]{0.00,0.00,1.00}{##1}}}
\expandafter\def\csname PY@tok@nc\endcsname{\let\PY@bf=\textbf\def\PY@tc##1{\textcolor[rgb]{0.00,0.00,1.00}{##1}}}
\expandafter\def\csname PY@tok@nn\endcsname{\let\PY@bf=\textbf\def\PY@tc##1{\textcolor[rgb]{0.00,0.00,1.00}{##1}}}
\expandafter\def\csname PY@tok@ne\endcsname{\let\PY@bf=\textbf\def\PY@tc##1{\textcolor[rgb]{0.82,0.25,0.23}{##1}}}
\expandafter\def\csname PY@tok@nv\endcsname{\def\PY@tc##1{\textcolor[rgb]{0.10,0.09,0.49}{##1}}}
\expandafter\def\csname PY@tok@no\endcsname{\def\PY@tc##1{\textcolor[rgb]{0.53,0.00,0.00}{##1}}}
\expandafter\def\csname PY@tok@nl\endcsname{\def\PY@tc##1{\textcolor[rgb]{0.63,0.63,0.00}{##1}}}
\expandafter\def\csname PY@tok@ni\endcsname{\let\PY@bf=\textbf\def\PY@tc##1{\textcolor[rgb]{0.60,0.60,0.60}{##1}}}
\expandafter\def\csname PY@tok@na\endcsname{\def\PY@tc##1{\textcolor[rgb]{0.49,0.56,0.16}{##1}}}
\expandafter\def\csname PY@tok@nt\endcsname{\let\PY@bf=\textbf\def\PY@tc##1{\textcolor[rgb]{0.00,0.50,0.00}{##1}}}
\expandafter\def\csname PY@tok@nd\endcsname{\def\PY@tc##1{\textcolor[rgb]{0.67,0.13,1.00}{##1}}}
\expandafter\def\csname PY@tok@s\endcsname{\def\PY@tc##1{\textcolor[rgb]{0.73,0.13,0.13}{##1}}}
\expandafter\def\csname PY@tok@sd\endcsname{\let\PY@it=\textit\def\PY@tc##1{\textcolor[rgb]{0.73,0.13,0.13}{##1}}}
\expandafter\def\csname PY@tok@si\endcsname{\let\PY@bf=\textbf\def\PY@tc##1{\textcolor[rgb]{0.73,0.40,0.53}{##1}}}
\expandafter\def\csname PY@tok@se\endcsname{\let\PY@bf=\textbf\def\PY@tc##1{\textcolor[rgb]{0.73,0.40,0.13}{##1}}}
\expandafter\def\csname PY@tok@sr\endcsname{\def\PY@tc##1{\textcolor[rgb]{0.73,0.40,0.53}{##1}}}
\expandafter\def\csname PY@tok@ss\endcsname{\def\PY@tc##1{\textcolor[rgb]{0.10,0.09,0.49}{##1}}}
\expandafter\def\csname PY@tok@sx\endcsname{\def\PY@tc##1{\textcolor[rgb]{0.00,0.50,0.00}{##1}}}
\expandafter\def\csname PY@tok@m\endcsname{\def\PY@tc##1{\textcolor[rgb]{0.40,0.40,0.40}{##1}}}
\expandafter\def\csname PY@tok@gh\endcsname{\let\PY@bf=\textbf\def\PY@tc##1{\textcolor[rgb]{0.00,0.00,0.50}{##1}}}
\expandafter\def\csname PY@tok@gu\endcsname{\let\PY@bf=\textbf\def\PY@tc##1{\textcolor[rgb]{0.50,0.00,0.50}{##1}}}
\expandafter\def\csname PY@tok@gd\endcsname{\def\PY@tc##1{\textcolor[rgb]{0.63,0.00,0.00}{##1}}}
\expandafter\def\csname PY@tok@gi\endcsname{\def\PY@tc##1{\textcolor[rgb]{0.00,0.63,0.00}{##1}}}
\expandafter\def\csname PY@tok@gr\endcsname{\def\PY@tc##1{\textcolor[rgb]{1.00,0.00,0.00}{##1}}}
\expandafter\def\csname PY@tok@ge\endcsname{\let\PY@it=\textit}
\expandafter\def\csname PY@tok@gs\endcsname{\let\PY@bf=\textbf}
\expandafter\def\csname PY@tok@gp\endcsname{\let\PY@bf=\textbf\def\PY@tc##1{\textcolor[rgb]{0.00,0.00,0.50}{##1}}}
\expandafter\def\csname PY@tok@go\endcsname{\def\PY@tc##1{\textcolor[rgb]{0.53,0.53,0.53}{##1}}}
\expandafter\def\csname PY@tok@gt\endcsname{\def\PY@tc##1{\textcolor[rgb]{0.00,0.27,0.87}{##1}}}
\expandafter\def\csname PY@tok@err\endcsname{\def\PY@bc##1{\setlength{\fboxsep}{0pt}\fcolorbox[rgb]{1.00,0.00,0.00}{1,1,1}{\strut ##1}}}
\expandafter\def\csname PY@tok@kc\endcsname{\let\PY@bf=\textbf\def\PY@tc##1{\textcolor[rgb]{0.00,0.50,0.00}{##1}}}
\expandafter\def\csname PY@tok@kd\endcsname{\let\PY@bf=\textbf\def\PY@tc##1{\textcolor[rgb]{0.00,0.50,0.00}{##1}}}
\expandafter\def\csname PY@tok@kn\endcsname{\let\PY@bf=\textbf\def\PY@tc##1{\textcolor[rgb]{0.00,0.50,0.00}{##1}}}
\expandafter\def\csname PY@tok@kr\endcsname{\let\PY@bf=\textbf\def\PY@tc##1{\textcolor[rgb]{0.00,0.50,0.00}{##1}}}
\expandafter\def\csname PY@tok@bp\endcsname{\def\PY@tc##1{\textcolor[rgb]{0.00,0.50,0.00}{##1}}}
\expandafter\def\csname PY@tok@fm\endcsname{\def\PY@tc##1{\textcolor[rgb]{0.00,0.00,1.00}{##1}}}
\expandafter\def\csname PY@tok@vc\endcsname{\def\PY@tc##1{\textcolor[rgb]{0.10,0.09,0.49}{##1}}}
\expandafter\def\csname PY@tok@vg\endcsname{\def\PY@tc##1{\textcolor[rgb]{0.10,0.09,0.49}{##1}}}
\expandafter\def\csname PY@tok@vi\endcsname{\def\PY@tc##1{\textcolor[rgb]{0.10,0.09,0.49}{##1}}}
\expandafter\def\csname PY@tok@vm\endcsname{\def\PY@tc##1{\textcolor[rgb]{0.10,0.09,0.49}{##1}}}
\expandafter\def\csname PY@tok@sa\endcsname{\def\PY@tc##1{\textcolor[rgb]{0.73,0.13,0.13}{##1}}}
\expandafter\def\csname PY@tok@sb\endcsname{\def\PY@tc##1{\textcolor[rgb]{0.73,0.13,0.13}{##1}}}
\expandafter\def\csname PY@tok@sc\endcsname{\def\PY@tc##1{\textcolor[rgb]{0.73,0.13,0.13}{##1}}}
\expandafter\def\csname PY@tok@dl\endcsname{\def\PY@tc##1{\textcolor[rgb]{0.73,0.13,0.13}{##1}}}
\expandafter\def\csname PY@tok@s2\endcsname{\def\PY@tc##1{\textcolor[rgb]{0.73,0.13,0.13}{##1}}}
\expandafter\def\csname PY@tok@sh\endcsname{\def\PY@tc##1{\textcolor[rgb]{0.73,0.13,0.13}{##1}}}
\expandafter\def\csname PY@tok@s1\endcsname{\def\PY@tc##1{\textcolor[rgb]{0.73,0.13,0.13}{##1}}}
\expandafter\def\csname PY@tok@mb\endcsname{\def\PY@tc##1{\textcolor[rgb]{0.40,0.40,0.40}{##1}}}
\expandafter\def\csname PY@tok@mf\endcsname{\def\PY@tc##1{\textcolor[rgb]{0.40,0.40,0.40}{##1}}}
\expandafter\def\csname PY@tok@mh\endcsname{\def\PY@tc##1{\textcolor[rgb]{0.40,0.40,0.40}{##1}}}
\expandafter\def\csname PY@tok@mi\endcsname{\def\PY@tc##1{\textcolor[rgb]{0.40,0.40,0.40}{##1}}}
\expandafter\def\csname PY@tok@il\endcsname{\def\PY@tc##1{\textcolor[rgb]{0.40,0.40,0.40}{##1}}}
\expandafter\def\csname PY@tok@mo\endcsname{\def\PY@tc##1{\textcolor[rgb]{0.40,0.40,0.40}{##1}}}
\expandafter\def\csname PY@tok@ch\endcsname{\let\PY@it=\textit\def\PY@tc##1{\textcolor[rgb]{0.25,0.50,0.50}{##1}}}
\expandafter\def\csname PY@tok@cm\endcsname{\let\PY@it=\textit\def\PY@tc##1{\textcolor[rgb]{0.25,0.50,0.50}{##1}}}
\expandafter\def\csname PY@tok@cpf\endcsname{\let\PY@it=\textit\def\PY@tc##1{\textcolor[rgb]{0.25,0.50,0.50}{##1}}}
\expandafter\def\csname PY@tok@c1\endcsname{\let\PY@it=\textit\def\PY@tc##1{\textcolor[rgb]{0.25,0.50,0.50}{##1}}}
\expandafter\def\csname PY@tok@cs\endcsname{\let\PY@it=\textit\def\PY@tc##1{\textcolor[rgb]{0.25,0.50,0.50}{##1}}}

\def\PYZbs{\char`\\}
\def\PYZus{\char`\_}
\def\PYZob{\char`\{}
\def\PYZcb{\char`\}}
\def\PYZca{\char`\^}
\def\PYZam{\char`\&}
\def\PYZlt{\char`\<}
\def\PYZgt{\char`\>}
\def\PYZsh{\char`\#}
\def\PYZpc{\char`\%}
\def\PYZdl{\char`\$}
\def\PYZhy{\char`\-}
\def\PYZsq{\char`\'}
\def\PYZdq{\char`\"}
\def\PYZti{\char`\~}
% for compatibility with earlier versions
\def\PYZat{@}
\def\PYZlb{[}
\def\PYZrb{]}
\makeatother


    % For linebreaks inside Verbatim environment from package fancyvrb. 
    \makeatletter
        \newbox\Wrappedcontinuationbox 
        \newbox\Wrappedvisiblespacebox 
        \newcommand*\Wrappedvisiblespace {\textcolor{red}{\textvisiblespace}} 
        \newcommand*\Wrappedcontinuationsymbol {\textcolor{red}{\llap{\tiny$\m@th\hookrightarrow$}}} 
        \newcommand*\Wrappedcontinuationindent {3ex } 
        \newcommand*\Wrappedafterbreak {\kern\Wrappedcontinuationindent\copy\Wrappedcontinuationbox} 
        % Take advantage of the already applied Pygments mark-up to insert 
        % potential linebreaks for TeX processing. 
        %        {, <, #, %, $, ' and ": go to next line. 
        %        _, }, ^, &, >, - and ~: stay at end of broken line. 
        % Use of \textquotesingle for straight quote. 
        \newcommand*\Wrappedbreaksatspecials {% 
            \def\PYGZus{\discretionary{\char`\_}{\Wrappedafterbreak}{\char`\_}}% 
            \def\PYGZob{\discretionary{}{\Wrappedafterbreak\char`\{}{\char`\{}}% 
            \def\PYGZcb{\discretionary{\char`\}}{\Wrappedafterbreak}{\char`\}}}% 
            \def\PYGZca{\discretionary{\char`\^}{\Wrappedafterbreak}{\char`\^}}% 
            \def\PYGZam{\discretionary{\char`\&}{\Wrappedafterbreak}{\char`\&}}% 
            \def\PYGZlt{\discretionary{}{\Wrappedafterbreak\char`\<}{\char`\<}}% 
            \def\PYGZgt{\discretionary{\char`\>}{\Wrappedafterbreak}{\char`\>}}% 
            \def\PYGZsh{\discretionary{}{\Wrappedafterbreak\char`\#}{\char`\#}}% 
            \def\PYGZpc{\discretionary{}{\Wrappedafterbreak\char`\%}{\char`\%}}% 
            \def\PYGZdl{\discretionary{}{\Wrappedafterbreak\char`\$}{\char`\$}}% 
            \def\PYGZhy{\discretionary{\char`\-}{\Wrappedafterbreak}{\char`\-}}% 
            \def\PYGZsq{\discretionary{}{\Wrappedafterbreak\textquotesingle}{\textquotesingle}}% 
            \def\PYGZdq{\discretionary{}{\Wrappedafterbreak\char`\"}{\char`\"}}% 
            \def\PYGZti{\discretionary{\char`\~}{\Wrappedafterbreak}{\char`\~}}% 
        } 
        % Some characters . , ; ? ! / are not pygmentized. 
        % This macro makes them "active" and they will insert potential linebreaks 
        \newcommand*\Wrappedbreaksatpunct {% 
            \lccode`\~`\.\lowercase{\def~}{\discretionary{\hbox{\char`\.}}{\Wrappedafterbreak}{\hbox{\char`\.}}}% 
            \lccode`\~`\,\lowercase{\def~}{\discretionary{\hbox{\char`\,}}{\Wrappedafterbreak}{\hbox{\char`\,}}}% 
            \lccode`\~`\;\lowercase{\def~}{\discretionary{\hbox{\char`\;}}{\Wrappedafterbreak}{\hbox{\char`\;}}}% 
            \lccode`\~`\:\lowercase{\def~}{\discretionary{\hbox{\char`\:}}{\Wrappedafterbreak}{\hbox{\char`\:}}}% 
            \lccode`\~`\?\lowercase{\def~}{\discretionary{\hbox{\char`\?}}{\Wrappedafterbreak}{\hbox{\char`\?}}}% 
            \lccode`\~`\!\lowercase{\def~}{\discretionary{\hbox{\char`\!}}{\Wrappedafterbreak}{\hbox{\char`\!}}}% 
            \lccode`\~`\/\lowercase{\def~}{\discretionary{\hbox{\char`\/}}{\Wrappedafterbreak}{\hbox{\char`\/}}}% 
            \catcode`\.\active
            \catcode`\,\active 
            \catcode`\;\active
            \catcode`\:\active
            \catcode`\?\active
            \catcode`\!\active
            \catcode`\/\active 
            \lccode`\~`\~ 	
        }
    \makeatother

    \let\OriginalVerbatim=\Verbatim
    \makeatletter
    \renewcommand{\Verbatim}[1][1]{%
        %\parskip\z@skip
        \sbox\Wrappedcontinuationbox {\Wrappedcontinuationsymbol}%
        \sbox\Wrappedvisiblespacebox {\FV@SetupFont\Wrappedvisiblespace}%
        \def\FancyVerbFormatLine ##1{\hsize\linewidth
            \vtop{\raggedright\hyphenpenalty\z@\exhyphenpenalty\z@
                \doublehyphendemerits\z@\finalhyphendemerits\z@
                \strut ##1\strut}%
        }%
        % If the linebreak is at a space, the latter will be displayed as visible
        % space at end of first line, and a continuation symbol starts next line.
        % Stretch/shrink are however usually zero for typewriter font.
        \def\FV@Space {%
            \nobreak\hskip\z@ plus\fontdimen3\font minus\fontdimen4\font
            \discretionary{\copy\Wrappedvisiblespacebox}{\Wrappedafterbreak}
            {\kern\fontdimen2\font}%
        }%
        
        % Allow breaks at special characters using \PYG... macros.
        \Wrappedbreaksatspecials
        % Breaks at punctuation characters . , ; ? ! and / need catcode=\active 	
        \OriginalVerbatim[#1,codes*=\Wrappedbreaksatpunct]%
    }
    \makeatother

    % Exact colors from NB
    \definecolor{incolor}{HTML}{303F9F}
    \definecolor{outcolor}{HTML}{D84315}
    \definecolor{cellborder}{HTML}{CFCFCF}
    \definecolor{cellbackground}{HTML}{F7F7F7}
    
    % prompt
    \makeatletter
    \newcommand{\boxspacing}{\kern\kvtcb@left@rule\kern\kvtcb@boxsep}
    \makeatother
    \newcommand{\prompt}[4]{
        \ttfamily\llap{{\color{#2}[#3]:\hspace{3pt}#4}}\vspace{-\baselineskip}
    }
    

    
    % Prevent overflowing lines due to hard-to-break entities
    \sloppy 
    % Setup hyperref package
    \hypersetup{
      breaklinks=true,  % so long urls are correctly broken across lines
      colorlinks=true,
      urlcolor=urlcolor,
      linkcolor=linkcolor,
      citecolor=citecolor,
      }
    % Slightly bigger margins than the latex defaults
    
    \geometry{verbose,tmargin=1in,bmargin=1in,lmargin=1in,rmargin=1in}
    
    

\begin{document}

\maketitle

\newpage

\section*{Code}

All code and data you can found in our repo \url{https://github.com/kventinel/hse-ml-project-mnist}

\section*{Data}

For our project we get \href{http://yann.lecun.com/exdb/mnist/}{MNIST} dataset, that consists from $60000$ images of digits from $0$ to $9$ in train and $10000$ in test. From this images we randomly choose $1000$ images in train and $1000$ images in test. For each image in original dataset presents $28 \times 28$ features in range from $0$ to $255$ -- value of each pixel of image.

    \begin{center}
    \adjustimage{max size={0.9\linewidth}{0.9\paperheight}}{report/mnist.png}
    \end{center}


We preprocess this image and get next features:
\begin{itemize}
            \item count -- count of nonzero pixels in image
            \item mean -- average value of pixels in image
            \item vert\_symmetry -- difference between mean of pixels in top half of image and bottom half of image
            \item hor\_symmetry -- difference between mean of pixels in right half of image and bottom half of image
            \item vert\_mass\_center -- weighted mean of pixels by this \href{https://en.wikipedia.org/wiki/Center_of_mass}{equation}, where wight of pixel -- it is index of row with this pixel
            \item hor\_mass\_center -- weighted mean of pixels by this \href{https://en.wikipedia.org/wiki/Center_of_mass}{equation}, where wight of pixel -- it is index of column with this pixel
\end{itemize}
And 3 features received using filters \href{https://en.wikipedia.org/wiki/Viola–Jones_object_detection_framework}{Viola-Jones}:
\begin{itemize}
        \item vert\_viola -- difference between (mean of pixels in top 7 and bottom 7 rows) and (mean of pixels in center 14 rows)
        \item hor\_viola -- difference between (mean of pixels in left 7 and right 7 columns) and (mean of pixels in center 14 columns)
        \item all\_viola -- difference between (mean of pixels in top left quarter and right bottom quarter) and (mean of pixels in top right quarter and bottom left quarter)
\end{itemize}

We have got this dataset, because it's very common dataset for all machine learning courses and articles. And we were interesting to know more different facts about patterns in this dataset.

\begin{tcolorbox}[breakable, size=fbox, boxrule=1pt, pad at break*=1mm,colback=cellbackground, colframe=cellborder]
\prompt{In}{incolor}{2}{\boxspacing}
\begin{Verbatim}[commandchars=\\\{\}]
\PY{n}{train} \PY{o}{=} \PY{n}{pd}\PY{o}{.}\PY{n}{read\PYZus{}csv}\PY{p}{(}\PY{n}{constants}\PY{o}{.}\PY{n}{TRAIN}\PY{p}{)}
\PY{n}{test} \PY{o}{=} \PY{n}{pd}\PY{o}{.}\PY{n}{read\PYZus{}csv}\PY{p}{(}\PY{n}{constants}\PY{o}{.}\PY{n}{TEST}\PY{p}{)}
\end{Verbatim}
\end{tcolorbox}

    \begin{tcolorbox}[breakable, size=fbox, boxrule=1pt, pad at break*=1mm,colback=cellbackground, colframe=cellborder]
\prompt{In}{incolor}{3}{\boxspacing}
\begin{Verbatim}[commandchars=\\\{\}]
\PY{n}{train}
\end{Verbatim}
\end{tcolorbox}

            \begin{tcolorbox}[breakable, size=fbox, boxrule=.5pt, pad at break*=1mm, opacityfill=0]
\prompt{Out}{outcolor}{3}{\boxspacing}
\begin{Verbatim}[commandchars=\\\{\}]
        mean    count  vert\_symmetry  hor\_symmetry          vert\_mass\_center  \textbackslash{}
0  26.538265      122      10.081633    -18.117347              14.299144
1  40.753827      155      -2.125000    -11.864796              13.984069
2  41.187500      176       0.323980     -4.267857              14.086804
3  38.184949      155      -2.043367     -2.477041              14.146140
4  48.011480      204      -6.772959    -21.385204              13.838341

         hor\_mass\_center  vert\_viola  hor\_viola  all\_viola  label
0           14.299385   -32.581633  -42.862245   19.836735      7
1           14.207067   -23.267857  -77.915816    0.829082      3
2           13.838004   -35.079082  -76.956633  -24.625000      8
3           13.564352   -48.140306  -76.369898  -13.706633      9
4           14.332457   -21.022959  -74.385204  -24.446429      3

[1000 rows x 12 columns]
\end{Verbatim}

\end{tcolorbox}





\section*{K-means}

    Let's firstly define functions for scaling the data and fitting KMeans
clustering. First function returns us KMeans criterion with clustering
labels based on 10 random intializations. We use sklearn implementation
which optimexes inertia as the fuctional.

    \begin{tcolorbox}[breakable, size=fbox, boxrule=1pt, pad at break*=1mm,colback=cellbackground, colframe=cellborder]
\prompt{In}{incolor}{6}{\boxspacing}
\begin{Verbatim}[commandchars=\\\{\}]
\PY{k}{def} \PY{n+nf}{fit\PYZus{}kmeans}\PY{p}{(}\PY{n}{X}\PY{p}{,} \PY{n}{features}\PY{p}{,} \PY{n}{n\PYZus{}clusters}\PY{p}{,} \PY{n}{n\PYZus{}init}\PY{p}{)}\PY{p}{:}
    \PY{n}{kmeans\PYZus{}criterion} \PY{o}{=} \PY{p}{[}\PY{p}{]}
    \PY{n}{clustering\PYZus{}results} \PY{o}{=} \PY{p}{[}\PY{p}{]}
    \PY{k}{for} \PY{n}{\PYZus{}} \PY{o+ow}{in} \PY{n+nb}{range}\PY{p}{(}\PY{n}{n\PYZus{}init}\PY{p}{)}\PY{p}{:}
        \PY{n}{clusterer} \PY{o}{=} \PY{n}{KMeans}\PY{p}{(}\PY{n}{n\PYZus{}clusters}\PY{o}{=}\PY{n}{n\PYZus{}clusters}\PY{p}{,} \PY{n}{n\PYZus{}init}\PY{o}{=}\PY{l+m+mi}{1}\PY{p}{)}
        \PY{n}{predicted\PYZus{}labels} \PY{o}{=} \PY{n}{clusterer}\PY{o}{.}\PY{n}{fit\PYZus{}predict}\PY{p}{(}\PY{n}{X}\PY{p}{[}\PY{n}{features}\PY{p}{]}\PY{p}{)}
        \PY{n}{clustering\PYZus{}results}\PY{o}{.}\PY{n}{append}\PY{p}{(}\PY{n}{predicted\PYZus{}labels}\PY{p}{)}
        \PY{n}{kmeans\PYZus{}criterion}\PY{o}{.}\PY{n}{append}\PY{p}{(}\PY{n}{clusterer}\PY{o}{.}\PY{n}{inertia\PYZus{}}\PY{p}{)}
    \PY{k}{return} \PY{n}{kmeans\PYZus{}criterion}\PY{p}{,} \PY{n}{clustering\PYZus{}results}\PY{p}{[}\PY{n}{np}\PY{o}{.}\PY{n}{argmin}\PY{p}{(}\PY{n}{kmeans\PYZus{}criterion}\PY{p}{)}\PY{p}{]}
\end{Verbatim}
\end{tcolorbox}

    \begin{tcolorbox}[breakable, size=fbox, boxrule=1pt, pad at break*=1mm,colback=cellbackground, colframe=cellborder]
\prompt{In}{incolor}{7}{\boxspacing}
\begin{Verbatim}[commandchars=\\\{\}]
\PY{k}{def} \PY{n+nf}{scale\PYZus{}data}\PY{p}{(}\PY{n}{X}\PY{p}{)}\PY{p}{:}
    \PY{n}{scaler} \PY{o}{=} \PY{n}{MinMaxScaler}\PY{p}{(}\PY{p}{)}
    \PY{n}{X\PYZus{}scaled} \PY{o}{=} \PY{n}{pd}\PY{o}{.}\PY{n}{DataFrame}\PY{p}{(}
        \PY{n}{scaler}\PY{o}{.}\PY{n}{fit\PYZus{}transform}\PY{p}{(}\PY{n}{X}\PY{p}{)}\PY{p}{,}
        \PY{n}{columns} \PY{o}{=} \PY{n}{X}\PY{o}{.}\PY{n}{columns}
    \PY{p}{)}
    
    \PY{k}{return} \PY{n}{X\PYZus{}scaled}
    
\end{Verbatim}
\end{tcolorbox}

    Now we could read the data and proceed to feature normalizing. We will
normalize only continous features using MinMaxScaler. This scaler
transform feature range to {[}0, 1{]} using its minimum and maximum
value.

    \begin{tcolorbox}[breakable, size=fbox, boxrule=1pt, pad at break*=1mm,colback=cellbackground, colframe=cellborder]
\prompt{In}{incolor}{8}{\boxspacing}
\begin{Verbatim}[commandchars=\\\{\}]
\PY{n}{clustering\PYZus{}features} \PY{o}{=} \PY{p}{[}\PY{l+s+s1}{\PYZsq{}}\PY{l+s+s1}{all\PYZus{}viola}\PY{l+s+s1}{\PYZsq{}}\PY{p}{,} \PY{l+s+s1}{\PYZsq{}}\PY{l+s+s1}{count}\PY{l+s+s1}{\PYZsq{}}\PY{p}{,} \PY{l+s+s1}{\PYZsq{}}\PY{l+s+s1}{vert\PYZus{}mass\PYZus{}center}\PY{l+s+s1}{\PYZsq{}}\PY{p}{,} \PY{l+s+s1}{\PYZsq{}}\PY{l+s+s1}{vert\PYZus{}symmetry}\PY{l+s+s1}{\PYZsq{}}\PY{p}{,} \PY{l+s+s1}{\PYZsq{}}\PY{l+s+s1}{hor\PYZus{}mass\PYZus{}center}\PY{l+s+s1}{\PYZsq{}}\PY{p}{]}
\PY{n}{clustering\PYZus{}features\PYZus{}scale} \PY{o}{=} \PY{p}{[}\PY{l+s+s1}{\PYZsq{}}\PY{l+s+s1}{all\PYZus{}viola}\PY{l+s+s1}{\PYZsq{}}\PY{p}{,} \PY{l+s+s1}{\PYZsq{}}\PY{l+s+s1}{vert\PYZus{}mass\PYZus{}center}\PY{l+s+s1}{\PYZsq{}}\PY{p}{,} \PY{l+s+s1}{\PYZsq{}}\PY{l+s+s1}{vert\PYZus{}symmetry}\PY{l+s+s1}{\PYZsq{}}\PY{p}{,} \PY{l+s+s1}{\PYZsq{}}\PY{l+s+s1}{hor\PYZus{}mass\PYZus{}center}\PY{l+s+s1}{\PYZsq{}}\PY{p}{]}

\PY{n}{train} \PY{o}{=} \PY{n}{pd}\PY{o}{.}\PY{n}{read\PYZus{}csv}\PY{p}{(}\PY{n}{constants}\PY{o}{.}\PY{n}{TRAIN}\PY{p}{,} 
                    \PY{n}{usecols} \PY{o}{=} \PY{p}{[}\PY{l+s+s1}{\PYZsq{}}\PY{l+s+s1}{all\PYZus{}viola}\PY{l+s+s1}{\PYZsq{}}\PY{p}{,} \PY{l+s+s1}{\PYZsq{}}\PY{l+s+s1}{count}\PY{l+s+s1}{\PYZsq{}}\PY{p}{,} \PY{l+s+s1}{\PYZsq{}}\PY{l+s+s1}{hor\PYZus{}mass\PYZus{}center}\PY{l+s+s1}{\PYZsq{}}\PY{p}{,} \PY{l+s+s1}{\PYZsq{}}\PY{l+s+s1}{hor\PYZus{}symmetry}\PY{l+s+s1}{\PYZsq{}}\PY{p}{,}
                               \PY{l+s+s1}{\PYZsq{}}\PY{l+s+s1}{hor\PYZus{}viola}\PY{l+s+s1}{\PYZsq{}}\PY{p}{,} \PY{l+s+s1}{\PYZsq{}}\PY{l+s+s1}{idx}\PY{l+s+s1}{\PYZsq{}}\PY{p}{,} \PY{l+s+s1}{\PYZsq{}}\PY{l+s+s1}{label}\PY{l+s+s1}{\PYZsq{}}\PY{p}{,} \PY{l+s+s1}{\PYZsq{}}\PY{l+s+s1}{mean}\PY{l+s+s1}{\PYZsq{}}\PY{p}{,} \PY{l+s+s1}{\PYZsq{}}\PY{l+s+s1}{vert\PYZus{}mass\PYZus{}center}\PY{l+s+s1}{\PYZsq{}}\PY{p}{,}
                                \PY{l+s+s1}{\PYZsq{}}\PY{l+s+s1}{vert\PYZus{}symmetry}\PY{l+s+s1}{\PYZsq{}}\PY{p}{,} \PY{l+s+s1}{\PYZsq{}}\PY{l+s+s1}{vert\PYZus{}viola}\PY{l+s+s1}{\PYZsq{}}\PY{p}{]}\PY{p}{)}

\PY{n}{train\PYZus{}scaled} \PY{o}{=} \PY{n}{scale\PYZus{}data}\PY{p}{(}\PY{n}{train}\PY{p}{[}\PY{n}{clustering\PYZus{}features\PYZus{}scale}\PY{p}{]}\PY{p}{)}
\PY{n}{train\PYZus{}scaled}\PY{p}{[}\PY{l+s+s1}{\PYZsq{}}\PY{l+s+s1}{count}\PY{l+s+s1}{\PYZsq{}}\PY{p}{]} \PY{o}{=} \PY{n}{train}\PY{p}{[}\PY{l+s+s1}{\PYZsq{}}\PY{l+s+s1}{count}\PY{l+s+s1}{\PYZsq{}}\PY{p}{]}
\PY{n}{train\PYZus{}scaled}\PY{p}{[}\PY{l+s+s1}{\PYZsq{}}\PY{l+s+s1}{label}\PY{l+s+s1}{\PYZsq{}}\PY{p}{]} \PY{o}{=} \PY{n}{train}\PY{p}{[}\PY{l+s+s1}{\PYZsq{}}\PY{l+s+s1}{label}\PY{l+s+s1}{\PYZsq{}}\PY{p}{]}
\end{Verbatim}
\end{tcolorbox}

    After we normalized the feature, we proceed to choosing the most basic
features for clusterization. They are based only on general information
about pixels and also their vertical axis. These are simple features
that could tell us about different pixels on the images and the
orientation of the object. Moreover, we tried to exclude the linear
dependent feature in order to bring better generalization to the
clustering.

    \begin{tcolorbox}[breakable, size=fbox, boxrule=1pt, pad at break*=1mm,colback=cellbackground, colframe=cellborder]
\prompt{In}{incolor}{9}{\boxspacing}
\begin{Verbatim}[commandchars=\\\{\}]
\PY{n}{sns}\PY{o}{.}\PY{n}{pairplot}\PY{p}{(}\PY{n}{train}\PY{p}{[}\PY{n}{clustering\PYZus{}features}\PY{p}{]}\PY{p}{)}
\end{Verbatim}
\end{tcolorbox}

            \begin{tcolorbox}[breakable, size=fbox, boxrule=.5pt, pad at break*=1mm, opacityfill=0]
\prompt{Out}{outcolor}{9}{\boxspacing}
\begin{Verbatim}[commandchars=\\\{\}]
<seaborn.axisgrid.PairGrid at 0x134ac4710>
\end{Verbatim}
\end{tcolorbox}
        
    \begin{center}
    \adjustimage{max size={0.9\linewidth}{0.9\paperheight}}{other/output_7_1.png}
    \end{center}
    { \hspace*{\fill} \\}
    
    Now, after normalizing the data we could fit KMeans and anylize result.
We will look at features means and comprare them to the grand mean (the
mean of the all dataset):

    \begin{tcolorbox}[breakable, size=fbox, boxrule=1pt, pad at break*=1mm,colback=cellbackground, colframe=cellborder]
\prompt{In}{incolor}{10}{\boxspacing}
\begin{Verbatim}[commandchars=\\\{\}]
\PY{n}{kmeans\PYZus{}five\PYZus{}criterion}\PY{p}{,} \PY{n}{labels\PYZus{}five} \PY{o}{=} \PY{n}{fit\PYZus{}kmeans}\PY{p}{(}\PY{n}{train\PYZus{}scaled}\PY{p}{,} \PY{n}{clustering\PYZus{}features}\PY{p}{,} \PY{l+m+mi}{5}\PY{p}{,} \PY{l+m+mi}{10}\PY{p}{)}
\PY{n}{kmeans\PYZus{}nine\PYZus{}criterion}\PY{p}{,} \PY{n}{labels\PYZus{}nine} \PY{o}{=} \PY{n}{fit\PYZus{}kmeans}\PY{p}{(}\PY{n}{train\PYZus{}scaled}\PY{p}{,} \PY{n}{clustering\PYZus{}features}\PY{p}{,} \PY{l+m+mi}{9}\PY{p}{,} \PY{l+m+mi}{10}\PY{p}{)}

\PY{n}{data\PYZus{}mean} \PY{o}{=} \PY{n}{train}\PY{p}{[}\PY{n}{clustering\PYZus{}features} \PY{o}{+} \PY{p}{[}\PY{l+s+s1}{\PYZsq{}}\PY{l+s+s1}{label}\PY{l+s+s1}{\PYZsq{}}\PY{p}{]}\PY{p}{]}\PY{o}{.}\PY{n}{mean}\PY{p}{(}\PY{p}{)}
\PY{n}{data\PYZus{}mean}\PY{o}{.}\PY{n}{name} \PY{o}{=} \PY{l+s+s1}{\PYZsq{}}\PY{l+s+s1}{data\PYZus{}mean}\PY{l+s+s1}{\PYZsq{}}
\end{Verbatim}
\end{tcolorbox}

    \begin{tcolorbox}[breakable, size=fbox, boxrule=1pt, pad at break*=1mm,colback=cellbackground, colframe=cellborder]
\prompt{In}{incolor}{11}{\boxspacing}
\begin{Verbatim}[commandchars=\\\{\}]
\PY{n}{train}\PY{p}{[}\PY{l+s+s1}{\PYZsq{}}\PY{l+s+s1}{clusters\PYZus{}five}\PY{l+s+s1}{\PYZsq{}}\PY{p}{]} \PY{o}{=} \PY{n}{labels\PYZus{}five}
\PY{n}{results\PYZus{}five\PYZus{}clusters} \PY{o}{=} \PY{n}{train}\PY{o}{.}\PY{n}{groupby}\PY{p}{(}\PY{l+s+s1}{\PYZsq{}}\PY{l+s+s1}{clusters\PYZus{}five}\PY{l+s+s1}{\PYZsq{}}\PY{p}{)}\PY{p}{[}\PY{n}{clustering\PYZus{}features} \PY{o}{+} \PY{p}{[}\PY{l+s+s1}{\PYZsq{}}\PY{l+s+s1}{label}\PY{l+s+s1}{\PYZsq{}}\PY{p}{]}\PY{p}{]}\PY{o}{.}\PY{n}{mean}\PY{p}{(}\PY{p}{)}
\PY{n}{results\PYZus{}five\PYZus{}clusters} \PY{o}{=} \PY{n}{results\PYZus{}five\PYZus{}clusters}\PY{o}{.}\PY{n}{append}\PY{p}{(}\PY{n}{data\PYZus{}mean}\PY{p}{)}
\PY{n}{results\PYZus{}five\PYZus{}clusters}
\end{Verbatim}
\end{tcolorbox}

            \begin{tcolorbox}[breakable, size=fbox, boxrule=.5pt, pad at break*=1mm, opacityfill=0]
\prompt{Out}{outcolor}{11}{\boxspacing}
\begin{Verbatim}[commandchars=\\\{\}]
               all\_viola       count  vert\_mass\_center  vert\_symmetry  \textbackslash{}
clusters\_five
0              -8.027731  191.709845         14.008225      -5.470339
1              -2.530407  123.515050         14.001630      -2.464183
2              -6.526573   82.371212         13.997653      -1.700854
3             -11.112132  235.852941         13.961299      -7.241409
4              -3.219222  156.629870         14.028761      -5.422144
data\_mean      -4.914597  149.084000         14.007992      -4.179515

               hor\_mass\_center     label
clusters\_five
0                    14.036303  3.968912
1                    14.020967  5.394649
2                    14.008330  2.628788
3                    13.997987  3.117647
4                    13.988845  5.032468
data\_mean            14.010802  4.488000
\end{Verbatim}
\end{tcolorbox}
        
    But before looking at the statistics let's look at the difference
between intializations:

    \begin{tcolorbox}[breakable, size=fbox, boxrule=1pt, pad at break*=1mm,colback=cellbackground, colframe=cellborder]
\prompt{In}{incolor}{12}{\boxspacing}
\begin{Verbatim}[commandchars=\\\{\}]
\PY{n}{pd}\PY{o}{.}\PY{n}{DataFrame}\PY{p}{(}\PY{n}{kmeans\PYZus{}five\PYZus{}criterion}\PY{p}{,} 
             \PY{n}{index} \PY{o}{=} \PY{p}{[}\PY{l+s+s2}{\PYZdq{}}\PY{l+s+s2}{run \PYZsh{}}\PY{l+s+s2}{\PYZdq{}} \PY{o}{+} \PY{n+nb}{str}\PY{p}{(}\PY{n}{x}\PY{p}{)} \PY{k}{for} \PY{n}{x} \PY{o+ow}{in} \PY{n+nb}{range}\PY{p}{(}\PY{l+m+mi}{1}\PY{p}{,} \PY{l+m+mi}{11}\PY{p}{)}\PY{p}{]}\PY{p}{,} 
             \PY{n}{columns} \PY{o}{=} \PY{p}{[}\PY{l+s+s1}{\PYZsq{}}\PY{l+s+s1}{inertia}\PY{l+s+s1}{\PYZsq{}}\PY{p}{]}
\PY{p}{)}
\end{Verbatim}
\end{tcolorbox}

            \begin{tcolorbox}[breakable, size=fbox, boxrule=.5pt, pad at break*=1mm, opacityfill=0]
\prompt{Out}{outcolor}{12}{\boxspacing}
\begin{Verbatim}[commandchars=\\\{\}]
               inertia
run \#1   202252.260511
run \#2   143905.693049
run \#3   145129.696990
run \#4   143696.083568
run \#5   142912.145003
run \#6   145129.696990
run \#7   145129.696990
run \#8   145129.696990
run \#9   145129.696990
run \#10  143725.213854
\end{Verbatim}
\end{tcolorbox}
        
    \begin{tcolorbox}[breakable, size=fbox, boxrule=1pt, pad at break*=1mm,colback=cellbackground, colframe=cellborder]
\prompt{In}{incolor}{13}{\boxspacing}
\begin{Verbatim}[commandchars=\\\{\}]
\PY{n}{pd}\PY{o}{.}\PY{n}{DataFrame}\PY{p}{(}\PY{n}{kmeans\PYZus{}nine\PYZus{}criterion}\PY{p}{,} 
             \PY{n}{index} \PY{o}{=} \PY{p}{[}\PY{l+s+s2}{\PYZdq{}}\PY{l+s+s2}{run \PYZsh{}}\PY{l+s+s2}{\PYZdq{}} \PY{o}{+} \PY{n+nb}{str}\PY{p}{(}\PY{n}{x}\PY{p}{)} \PY{k}{for} \PY{n}{x} \PY{o+ow}{in} \PY{n+nb}{range}\PY{p}{(}\PY{l+m+mi}{1}\PY{p}{,} \PY{l+m+mi}{11}\PY{p}{)}\PY{p}{]}\PY{p}{,} 
             \PY{n}{columns} \PY{o}{=} \PY{p}{[}\PY{l+s+s1}{\PYZsq{}}\PY{l+s+s1}{inertia}\PY{l+s+s1}{\PYZsq{}}\PY{p}{]}
\PY{p}{)}
\end{Verbatim}
\end{tcolorbox}

            \begin{tcolorbox}[breakable, size=fbox, boxrule=.5pt, pad at break*=1mm, opacityfill=0]
\prompt{Out}{outcolor}{13}{\boxspacing}
\begin{Verbatim}[commandchars=\\\{\}]
              inertia
run \#1   50980.675144
run \#2   51068.938199
run \#3   50280.957256
run \#4   53876.890407
run \#5   52484.961622
run \#6   52516.126284
run \#7   50213.154914
run \#8   50749.273014
run \#9   51096.792553
run \#10  51423.927640
\end{Verbatim}
\end{tcolorbox}
        
    According to the inertia the best runs are \#10 and \#3 for five and
nine clusters case respectively

    \begin{tcolorbox}[breakable, size=fbox, boxrule=1pt, pad at break*=1mm,colback=cellbackground, colframe=cellborder]
\prompt{In}{incolor}{14}{\boxspacing}
\begin{Verbatim}[commandchars=\\\{\}]
\PY{n}{rel\PYZus{}difference} \PY{o}{=} \PY{l+m+mi}{100} \PY{o}{*} \PY{p}{(}\PY{n}{results\PYZus{}five\PYZus{}clusters}\PY{o}{.}\PY{n}{iloc}\PY{p}{[}\PY{p}{:}\PY{o}{\PYZhy{}}\PY{l+m+mi}{1}\PY{p}{]} \PY{o}{\PYZhy{}} \PY{n}{results\PYZus{}five\PYZus{}clusters}\PY{o}{.}\PY{n}{loc}\PY{p}{[}\PY{l+s+s1}{\PYZsq{}}\PY{l+s+s1}{data\PYZus{}mean}\PY{l+s+s1}{\PYZsq{}}\PY{p}{]}\PY{p}{)} \PYZbs{}
                \PY{o}{/} \PY{n}{results\PYZus{}five\PYZus{}clusters}\PY{o}{.}\PY{n}{loc}\PY{p}{[}\PY{l+s+s1}{\PYZsq{}}\PY{l+s+s1}{data\PYZus{}mean}\PY{l+s+s1}{\PYZsq{}}\PY{p}{]}
\PY{n}{rel\PYZus{}difference}\PY{o}{.}\PY{n}{columns} \PY{o}{=} \PY{p}{[}\PY{n}{x} \PY{o}{+} \PY{l+s+s1}{\PYZsq{}}\PY{l+s+s1}{\PYZus{}diff, }\PY{l+s+s1}{\PYZpc{}}\PY{l+s+s1}{\PYZsq{}} \PY{k}{for} \PY{n}{x} \PY{o+ow}{in} \PY{n}{rel\PYZus{}difference}\PY{o}{.}\PY{n}{columns}\PY{p}{]}


\PY{n}{rel\PYZus{}difference} 
\end{Verbatim}
\end{tcolorbox}

            \begin{tcolorbox}[breakable, size=fbox, boxrule=.5pt, pad at break*=1mm, opacityfill=0]
\prompt{Out}{outcolor}{14}{\boxspacing}
\begin{Verbatim}[commandchars=\\\{\}]
               all\_viola\_diff, \%  count\_diff, \%  vert\_mass\_center\_diff, \%  \textbackslash{}
clusters\_five
0                      63.344642      28.591830                  0.001665
1                     -48.512411     -17.150700                 -0.045415
2                      32.799764     -44.748456                 -0.073803
3                     126.104653      58.201377                 -0.333330
4                     -34.496722       5.061489                  0.148267

               vert\_symmetry\_diff, \%  hor\_mass\_center\_diff, \%  label\_diff, \%
clusters\_five
0                          30.884541                 0.182008     -11.566134
1                         -41.041409                 0.072545      20.201623
2                         -59.304989                -0.017646     -41.426295
3                          73.259542                -0.091471     -30.533711
4                          29.731411                -0.156718      12.131630
\end{Verbatim}
\end{tcolorbox}
        
    \begin{tcolorbox}[breakable, size=fbox, boxrule=1pt, pad at break*=1mm,colback=cellbackground, colframe=cellborder]
\prompt{In}{incolor}{15}{\boxspacing}
\begin{Verbatim}[commandchars=\\\{\}]
\PY{n}{pd}\PY{o}{.}\PY{n}{crosstab}\PY{p}{(}\PY{n}{train}\PY{p}{[}\PY{l+s+s1}{\PYZsq{}}\PY{l+s+s1}{clusters\PYZus{}five}\PY{l+s+s1}{\PYZsq{}}\PY{p}{]}\PY{p}{,} \PY{n}{train}\PY{p}{[}\PY{l+s+s1}{\PYZsq{}}\PY{l+s+s1}{label}\PY{l+s+s1}{\PYZsq{}}\PY{p}{]}\PY{p}{)}
\end{Verbatim}
\end{tcolorbox}

            \begin{tcolorbox}[breakable, size=fbox, boxrule=.5pt, pad at break*=1mm, opacityfill=0]
\prompt{Out}{outcolor}{15}{\boxspacing}
\begin{Verbatim}[commandchars=\\\{\}]
label           0   1   2   3   4   5   6   7   8   9
clusters\_five
0              53   0  25  24   9  10  13   5  41  13
1               7  20  23  16  45  38  31  55  18  46
2               0  88   2   5   3   5   6  20   0   3
3              28   0   9   8   0   3   3   0  16   1
4              23   2  40  30  38  37  37  26  38  37
\end{Verbatim}
\end{tcolorbox}
        
    \begin{tcolorbox}[breakable, size=fbox, boxrule=1pt, pad at break*=1mm,colback=cellbackground, colframe=cellborder]
\prompt{In}{incolor}{16}{\boxspacing}
\begin{Verbatim}[commandchars=\\\{\}]
\PY{n}{train}\PY{p}{[}\PY{l+s+s1}{\PYZsq{}}\PY{l+s+s1}{clusters\PYZus{}nine}\PY{l+s+s1}{\PYZsq{}}\PY{p}{]} \PY{o}{=} \PY{n}{labels\PYZus{}nine}
\PY{n}{results\PYZus{}nine\PYZus{}clusters} \PY{o}{=} \PY{n}{train}\PY{o}{.}\PY{n}{groupby}\PY{p}{(}\PY{l+s+s1}{\PYZsq{}}\PY{l+s+s1}{clusters\PYZus{}nine}\PY{l+s+s1}{\PYZsq{}}\PY{p}{)}\PY{p}{[}\PY{n}{clustering\PYZus{}features} \PY{o}{+} \PY{p}{[}\PY{l+s+s1}{\PYZsq{}}\PY{l+s+s1}{label}\PY{l+s+s1}{\PYZsq{}}\PY{p}{]}\PY{p}{]}\PY{o}{.}\PY{n}{mean}\PY{p}{(}\PY{p}{)}
\PY{n}{results\PYZus{}nine\PYZus{}clusters} \PY{o}{=} \PY{n}{results\PYZus{}nine\PYZus{}clusters}\PY{o}{.}\PY{n}{append}\PY{p}{(}\PY{n}{data\PYZus{}mean}\PY{p}{)}
\PY{n}{results\PYZus{}nine\PYZus{}clusters}
\end{Verbatim}
\end{tcolorbox}

            \begin{tcolorbox}[breakable, size=fbox, boxrule=.5pt, pad at break*=1mm, opacityfill=0]
\prompt{Out}{outcolor}{16}{\boxspacing}
\begin{Verbatim}[commandchars=\\\{\}]
               all\_viola       count  vert\_mass\_center  vert\_symmetry  \textbackslash{}
clusters\_nine
0              -2.874304  150.898990         14.036684      -4.666048
1              -7.343899  194.550000         14.014254      -5.087096
2              -9.262894   75.923913         13.993732      -1.615600
3              -2.845576  106.726027         13.993887      -1.386847
4             -11.906353  249.419355         13.981966      -6.900428
5              -5.692571  170.487654         14.003582      -6.940271
6              -1.815851  130.761658         14.008533      -3.161600
7              11.522959  351.000000         13.589157      -4.002551
8              -9.935911  216.736842         13.986490      -5.993734
data\_mean      -4.914597  149.084000         14.007992      -4.179515

               hor\_mass\_center     label
clusters\_nine
0                    13.990923  5.186869
1                    14.035304  3.966667
2                    14.029493  1.891304
3                    14.033635  5.157534
4                    14.016093  3.451613
5                    13.992744  4.598765
6                    13.998676  5.352332
7                    14.487500  0.000000
8                    14.020768  3.035088
data\_mean            14.010802  4.488000
\end{Verbatim}
\end{tcolorbox}
        
    \begin{tcolorbox}[breakable, size=fbox, boxrule=1pt, pad at break*=1mm,colback=cellbackground, colframe=cellborder]
\prompt{In}{incolor}{17}{\boxspacing}
\begin{Verbatim}[commandchars=\\\{\}]
\PY{n}{abs\PYZus{}difference} \PY{o}{=} \PY{n}{results\PYZus{}nine\PYZus{}clusters}\PY{o}{.}\PY{n}{iloc}\PY{p}{[}\PY{p}{:}\PY{o}{\PYZhy{}}\PY{l+m+mi}{1}\PY{p}{]} \PY{o}{\PYZhy{}} \PY{n}{results\PYZus{}nine\PYZus{}clusters}\PY{o}{.}\PY{n}{loc}\PY{p}{[}\PY{l+s+s1}{\PYZsq{}}\PY{l+s+s1}{data\PYZus{}mean}\PY{l+s+s1}{\PYZsq{}}\PY{p}{]}
\PY{n}{abs\PYZus{}difference}\PY{o}{.}\PY{n}{columns} \PY{o}{=} \PY{p}{[}\PY{n}{x} \PY{o}{+} \PY{l+s+s1}{\PYZsq{}}\PY{l+s+s1}{\PYZus{}diff}\PY{l+s+s1}{\PYZsq{}} \PY{k}{for} \PY{n}{x} \PY{o+ow}{in} \PY{n}{abs\PYZus{}difference}\PY{o}{.}\PY{n}{columns}\PY{p}{]}

\PY{n}{abs\PYZus{}difference}
\end{Verbatim}
\end{tcolorbox}

            \begin{tcolorbox}[breakable, size=fbox, boxrule=.5pt, pad at break*=1mm, opacityfill=0]
\prompt{Out}{outcolor}{17}{\boxspacing}
\begin{Verbatim}[commandchars=\\\{\}]
               all\_viola\_diff  count\_diff  vert\_mass\_center\_diff  \textbackslash{}
clusters\_nine
0                    2.040293    1.814990               0.028692
1                   -2.429302   45.466000               0.006263
2                   -4.348297  -73.160087              -0.014259
3                    2.069021  -42.357973              -0.014105
4                   -6.991756  100.335355              -0.026025
5                   -0.777974   21.403654              -0.004410
6                    3.098746  -18.322342               0.000541
7                   16.437556  201.916000              -0.418834
8                   -5.021314   67.652842              -0.021502

               vert\_symmetry\_diff  hor\_mass\_center\_diff  label\_diff
clusters\_nine
0                       -0.486533             -0.019879    0.698869
1                       -0.907581              0.024502   -0.521333
2                        2.563915              0.018691   -2.596696
3                        2.792669              0.022832    0.669534
4                       -2.720913              0.005290   -1.036387
5                       -2.760756             -0.018059    0.110765
6                        1.017915             -0.012126    0.864332
7                        0.176964              0.476698   -4.488000
8                       -1.814219              0.009965   -1.452912
\end{Verbatim}
\end{tcolorbox}
        
    \begin{tcolorbox}[breakable, size=fbox, boxrule=1pt, pad at break*=1mm,colback=cellbackground, colframe=cellborder]
\prompt{In}{incolor}{18}{\boxspacing}
\begin{Verbatim}[commandchars=\\\{\}]
\PY{n}{rel\PYZus{}difference} \PY{o}{=} \PY{l+m+mi}{100} \PY{o}{*} \PY{p}{(}\PY{n}{results\PYZus{}nine\PYZus{}clusters}\PY{o}{.}\PY{n}{iloc}\PY{p}{[}\PY{p}{:}\PY{o}{\PYZhy{}}\PY{l+m+mi}{1}\PY{p}{]} \PY{o}{\PYZhy{}} \PY{n}{results\PYZus{}nine\PYZus{}clusters}\PY{o}{.}\PY{n}{loc}\PY{p}{[}\PY{l+s+s1}{\PYZsq{}}\PY{l+s+s1}{data\PYZus{}mean}\PY{l+s+s1}{\PYZsq{}}\PY{p}{]}\PY{p}{)} \PYZbs{}
                \PY{o}{/} \PY{n}{results\PYZus{}nine\PYZus{}clusters}\PY{o}{.}\PY{n}{loc}\PY{p}{[}\PY{l+s+s1}{\PYZsq{}}\PY{l+s+s1}{data\PYZus{}mean}\PY{l+s+s1}{\PYZsq{}}\PY{p}{]}
\PY{n}{rel\PYZus{}difference}\PY{o}{.}\PY{n}{columns} \PY{o}{=} \PY{p}{[}\PY{n}{x} \PY{o}{+} \PY{l+s+s1}{\PYZsq{}}\PY{l+s+s1}{\PYZus{}diff, }\PY{l+s+s1}{\PYZpc{}}\PY{l+s+s1}{\PYZsq{}} \PY{k}{for} \PY{n}{x} \PY{o+ow}{in} \PY{n}{rel\PYZus{}difference}\PY{o}{.}\PY{n}{columns}\PY{p}{]}


\PY{n}{rel\PYZus{}difference} 
\end{Verbatim}
\end{tcolorbox}

            \begin{tcolorbox}[breakable, size=fbox, boxrule=.5pt, pad at break*=1mm, opacityfill=0]
\prompt{Out}{outcolor}{18}{\boxspacing}
\begin{Verbatim}[commandchars=\\\{\}]
               all\_viola\_diff, \%  count\_diff, \%  vert\_mass\_center\_diff, \%  \textbackslash{}
clusters\_nine
0                     -41.514954       1.217428                  0.204824
1                      49.430338      30.496901                  0.044707
2                      88.477181     -49.073064                 -0.101795
3                     -42.099506     -28.412152                 -0.100694
4                     142.265093      67.301223                 -0.185789
5                      15.829856      14.356775                 -0.031484
6                     -63.051890     -12.289945                  0.003862
7                    -334.463972     135.437740                 -2.989967
8                     102.171436      45.379009                 -0.153495

               vert\_symmetry\_diff, \%  hor\_mass\_center\_diff, \%  label\_diff, \%
clusters\_nine
0                          11.640894                -0.141887      15.571940
1                          21.714977                 0.174879     -11.616162
2                         -61.344799                 0.133404     -57.858638
3                         -66.818006                 0.162963      14.918321
4                          65.101152                 0.037758     -23.092404
5                          66.054458                -0.128893       2.468035
6                         -24.354868                -0.086549      19.258726
7                          -4.234086                 3.402361    -100.000000
8                          43.407402                 0.071125     -32.373268
\end{Verbatim}
\end{tcolorbox}
        
    \begin{tcolorbox}[breakable, size=fbox, boxrule=1pt, pad at break*=1mm,colback=cellbackground, colframe=cellborder]
\prompt{In}{incolor}{19}{\boxspacing}
\begin{Verbatim}[commandchars=\\\{\}]
\PY{n}{pd}\PY{o}{.}\PY{n}{crosstab}\PY{p}{(}\PY{n}{train}\PY{p}{[}\PY{l+s+s1}{\PYZsq{}}\PY{l+s+s1}{clusters\PYZus{}nine}\PY{l+s+s1}{\PYZsq{}}\PY{p}{]}\PY{p}{,} \PY{n}{train}\PY{p}{[}\PY{l+s+s1}{\PYZsq{}}\PY{l+s+s1}{label}\PY{l+s+s1}{\PYZsq{}}\PY{p}{]}\PY{p}{)}
\end{Verbatim}
\end{tcolorbox}

            \begin{tcolorbox}[breakable, size=fbox, boxrule=.5pt, pad at break*=1mm, opacityfill=0]
\prompt{Out}{outcolor}{19}{\boxspacing}
\begin{Verbatim}[commandchars=\\\{\}]
label           0   1   2   3   4   5   6   7   8   9
clusters\_nine
0              11   2  21  20  30  24  24  18  19  29
1              35   0  14  16   1   8   9   2  27   8
2               0  76   0   2   0   2   2  10   0   0
3               0  26   6   8  18  13  13  39   4  19
4              14   0   2   2   0   1   2   0  10   0
5              20   0  29  16  16  14  17  10  27  13
6               7   6  19  11  30  28  22  26  14  30
7               1   0   0   0   0   0   0   0   0   0
8              23   0   8   8   0   3   1   1  12   1
\end{Verbatim}
\end{tcolorbox}
        
    Interpretation:

9 clusters case:

\begin{itemize}
\item
  Cluster five has the biggest difference with grand mean in feature
  ``count''. It means that in average case there are more black pixels
  in cluster five than in the whole sample.
\item
  Cluster zero has a lot of similar statistics compared to the data
  mean. For example, mean\_diff and vert\_mass\_center\_diff are 1.4 and
  0.12 percents respectively. It shows us that the distribution of
  different digits in cluster zero is almost the same as in the whole
  sample.
\item
  Cluster three has a big difference in all\_viola feature because there
  most popular digit in this clusters are zero and which have one of the
  biggest viola mean in the whole dataset.
\end{itemize}

5 clusters case:

This case is more interesting for analysis because we see a big relative
difference in almost every claster when comparing to the grand mean of
every feature. Let's see:

\begin{itemize}
\item
  Cluster two has the smallest mean of the feature named count. It could
  be easily explained by analyzing the most popular digit in the
  cluster. It's a digit 1, which almost doesn't have white pixels. That
  is why this cluster has the smallest mean of feature count.
\item
  Like in the 9 clusters case there is a cluster which has some features
  that does not differ from grand mean. For example, mean and
  vert\_mass\_center\_diff
\end{itemize}

Overall, the clusterization is very noisy for both cases. We could see
that by looking ath distribution of digits between the clusters. There
are always some clusters that contain each type of digits and they also
have very small relative difference in their mean. This clusterization
technique will be much better if we add raw features from the data or
more features created based on raw data.


\subsection*{Bootstrap}

    Firstly, let us define fucntion which computes both pivotal and
non-pivotal version of the bootstrap. It takes the following arguments:

\begin{itemize}
\item
  X - our data
\item
  feature - feature for which we computing bootstrap mean
\item
  cluster\_label - the name of the cluster feature where we store
  clusterinf results
\item
  first\_cluster - the first cluster number for bootstrap computation
\item
  first\_cluster - the second cluster number for bootstrap computation
\item
  pivotal - booalean variable which tells us if use pivotal version or
  not
\item
  :return: dictionary with three keys, each one corresponds to first
  cluster mean, second cluster mean or grand mean. The value is a list
  which contains left and right bounf for CI
\end{itemize}

    \begin{tcolorbox}[breakable, size=fbox, boxrule=1pt, pad at break*=1mm,colback=cellbackground, colframe=cellborder]
\prompt{In}{incolor}{20}{\boxspacing}
\begin{Verbatim}[commandchars=\\\{\}]
\PY{k}{def} \PY{n+nf}{bootstrap}\PY{p}{(}\PY{n}{X}\PY{p}{,} \PY{n}{feature}\PY{p}{,} \PY{n}{cluster\PYZus{}label}\PY{p}{,} \PY{n}{first\PYZus{}cluster}\PY{o}{=}\PY{k+kc}{None}\PY{p}{,} 
              \PY{n}{second\PYZus{}cluster}\PY{o}{=}\PY{k+kc}{None}\PY{p}{,} \PY{n}{pivotal}\PY{o}{=}\PY{k+kc}{True}\PY{p}{)}\PY{p}{:}
    \PY{n}{first\PYZus{}sample\PYZus{}means} \PY{o}{=} \PY{p}{[}\PY{p}{]}
    \PY{n}{second\PYZus{}sample\PYZus{}means} \PY{o}{=} \PY{p}{[}\PY{p}{]}
    \PY{n}{grand\PYZus{}sample\PYZus{}means} \PY{o}{=} \PY{p}{[}\PY{p}{]}
    \PY{k}{for} \PY{n}{\PYZus{}} \PY{o+ow}{in} \PY{n+nb}{range}\PY{p}{(}\PY{l+m+mi}{1000}\PY{p}{)}\PY{p}{:} \PY{c+c1}{\PYZsh{} bootstraping}
        \PY{k}{if} \PY{n}{first\PYZus{}cluster} \PY{o+ow}{is} \PY{o+ow}{not} \PY{k+kc}{None}\PY{p}{:}
            \PY{n}{first\PYZus{}sample\PYZus{}means}\PY{o}{.}\PY{n}{append}\PY{p}{(}
                \PY{n}{X}\PY{p}{[}\PY{n}{X}\PY{p}{[}\PY{n}{cluster\PYZus{}label}\PY{p}{]} \PY{o}{==} \PY{n}{first\PYZus{}cluster}\PY{p}{]}\PY{p}{[}\PY{n}{feature}\PY{p}{]} \PYZbs{}
                                                    \PY{o}{.}\PY{n}{sample}\PY{p}{(}\PY{l+m+mi}{100}\PY{p}{,} \PY{n}{replace}\PY{o}{=}\PY{k+kc}{True}\PY{p}{)} \PYZbs{}
                                                    \PY{o}{.}\PY{n}{mean}\PY{p}{(}\PY{p}{)}
            \PY{p}{)}
        \PY{k}{if} \PY{n}{second\PYZus{}cluster} \PY{o+ow}{is} \PY{o+ow}{not} \PY{k+kc}{None}\PY{p}{:}
            \PY{n}{second\PYZus{}sample\PYZus{}means}\PY{o}{.}\PY{n}{append}\PY{p}{(}
                \PY{n}{X}\PY{p}{[}\PY{n}{X}\PY{p}{[}\PY{n}{cluster\PYZus{}label}\PY{p}{]} \PY{o}{==} \PY{n}{second\PYZus{}cluster}\PY{p}{]}\PY{p}{[}\PY{n}{feature}\PY{p}{]}
                                                    \PY{o}{.}\PY{n}{sample}\PY{p}{(}\PY{l+m+mi}{100}\PY{p}{,} \PY{n}{replace}\PY{o}{=}\PY{k+kc}{True}\PY{p}{)} \PYZbs{}
                                                    \PY{o}{.}\PY{n}{mean}\PY{p}{(}\PY{p}{)}
            \PY{p}{)}
            
        \PY{n}{grand\PYZus{}sample\PYZus{}means}\PY{o}{.}\PY{n}{append}\PY{p}{(}
            \PY{n}{X}\PY{p}{[}\PY{n}{feature}\PY{p}{]}\PY{o}{.}\PY{n}{sample}\PY{p}{(}\PY{l+m+mi}{500}\PY{p}{)}\PY{o}{.}\PY{n}{mean}\PY{p}{(}\PY{p}{)}
        \PY{p}{)}
            
    \PY{k}{if} \PY{n}{pivotal}\PY{p}{:} 
        \PY{k}{if} \PY{n}{first\PYZus{}cluster} \PY{o+ow}{is} \PY{o+ow}{not} \PY{k+kc}{None}\PY{p}{:}
            \PY{n}{first\PYZus{}mean} \PY{o}{=} \PY{n}{np}\PY{o}{.}\PY{n}{mean}\PY{p}{(}\PY{n}{first\PYZus{}sample\PYZus{}means}\PY{p}{)}
            \PY{n}{first\PYZus{}std} \PY{o}{=} \PY{n}{np}\PY{o}{.}\PY{n}{std}\PY{p}{(}\PY{n}{first\PYZus{}sample\PYZus{}means}\PY{p}{)}
            
        \PY{k}{if} \PY{n}{second\PYZus{}cluster} \PY{o+ow}{is} \PY{o+ow}{not} \PY{k+kc}{None}\PY{p}{:}
            \PY{n}{second\PYZus{}mean} \PY{o}{=} \PY{n}{np}\PY{o}{.}\PY{n}{mean}\PY{p}{(}\PY{n}{second\PYZus{}sample\PYZus{}means}\PY{p}{)}
            \PY{n}{second\PYZus{}std} \PY{o}{=} \PY{n}{np}\PY{o}{.}\PY{n}{std}\PY{p}{(}\PY{n}{second\PYZus{}sample\PYZus{}means}\PY{p}{)}
        
        \PY{n}{grand\PYZus{}std} \PY{o}{=} \PY{n}{np}\PY{o}{.}\PY{n}{std}\PY{p}{(}\PY{n}{grand\PYZus{}sample\PYZus{}means}\PY{p}{)}
        \PY{n}{grand\PYZus{}mean} \PY{o}{=} \PY{n}{np}\PY{o}{.}\PY{n}{mean}\PY{p}{(}\PY{n}{grand\PYZus{}sample\PYZus{}means}\PY{p}{)}
        
        \PY{n+nb}{print}\PY{p}{(}\PY{n}{first\PYZus{}mean}\PY{p}{)}
        \PY{k}{return} \PY{p}{\PYZob{}}
            \PY{l+s+s1}{\PYZsq{}}\PY{l+s+s1}{first\PYZus{}CI}\PY{l+s+s1}{\PYZsq{}} \PY{p}{:} \PY{p}{[}\PY{n}{first\PYZus{}mean} \PY{o}{\PYZhy{}} \PY{l+m+mf}{1.96} \PY{o}{*} \PY{n}{first\PYZus{}std}\PY{p}{,} \PY{n}{first\PYZus{}mean} \PY{o}{+} \PY{l+m+mf}{1.96} \PY{o}{*} \PY{n}{first\PYZus{}std}\PY{p}{]}\PY{p}{,} 
            \PY{c+c1}{\PYZsh{} 1.96 is quantile of normal distribution}
            \PY{l+s+s1}{\PYZsq{}}\PY{l+s+s1}{second\PYZus{}CI}\PY{l+s+s1}{\PYZsq{}} \PY{p}{:} \PY{p}{[}\PY{n}{second\PYZus{}mean} \PY{o}{\PYZhy{}} \PY{l+m+mf}{1.96} \PY{o}{*} \PY{n}{second\PYZus{}std}\PY{p}{,} \PY{n}{second\PYZus{}mean} \PY{o}{+} \PY{l+m+mf}{1.96} \PY{o}{*} \PY{n}{second\PYZus{}std}\PY{p}{]}\PY{p}{,}
            \PY{l+s+s1}{\PYZsq{}}\PY{l+s+s1}{grand\PYZus{}CI}\PY{l+s+s1}{\PYZsq{}} \PY{p}{:} \PY{p}{[}\PY{n}{grand\PYZus{}mean} \PY{o}{\PYZhy{}} \PY{l+m+mf}{1.96} \PY{o}{*} \PY{n}{grand\PYZus{}std}\PY{p}{,} \PY{n}{grand\PYZus{}mean} \PY{o}{+} \PY{l+m+mf}{1.96} \PY{o}{*} \PY{n}{grand\PYZus{}std}\PY{p}{]}
        \PY{p}{\PYZcb{}}
    \PY{k}{else}\PY{p}{:}
        \PY{k}{if} \PY{n}{first\PYZus{}cluster} \PY{o+ow}{is} \PY{o+ow}{not} \PY{k+kc}{None}\PY{p}{:}
            \PY{n}{first\PYZus{}lb} \PY{o}{=} \PY{n}{np}\PY{o}{.}\PY{n}{percentile}\PY{p}{(}\PY{n}{first\PYZus{}sample\PYZus{}means}\PY{p}{,} \PY{l+m+mf}{2.5}\PY{p}{)}
            \PY{n}{first\PYZus{}rb} \PY{o}{=} \PY{n}{np}\PY{o}{.}\PY{n}{percentile}\PY{p}{(}\PY{n}{first\PYZus{}sample\PYZus{}means}\PY{p}{,} \PY{l+m+mf}{97.5}\PY{p}{)}
            
        \PY{k}{if} \PY{n}{second\PYZus{}cluster} \PY{o+ow}{is} \PY{o+ow}{not} \PY{k+kc}{None}\PY{p}{:}
            \PY{n}{second\PYZus{}lb} \PY{o}{=} \PY{n}{np}\PY{o}{.}\PY{n}{percentile}\PY{p}{(}\PY{n}{second\PYZus{}sample\PYZus{}means}\PY{p}{,} \PY{l+m+mf}{2.5}\PY{p}{)}
            \PY{n}{second\PYZus{}rb} \PY{o}{=} \PY{n}{np}\PY{o}{.}\PY{n}{percentile}\PY{p}{(}\PY{n}{second\PYZus{}sample\PYZus{}means}\PY{p}{,} \PY{l+m+mf}{97.5}\PY{p}{)}
        
        \PY{n}{grand\PYZus{}lb} \PY{o}{=} \PY{n}{np}\PY{o}{.}\PY{n}{percentile}\PY{p}{(}\PY{n}{grand\PYZus{}sample\PYZus{}means}\PY{p}{,} \PY{l+m+mf}{2.5}\PY{p}{)}
        \PY{n}{grand\PYZus{}rb} \PY{o}{=} \PY{n}{np}\PY{o}{.}\PY{n}{percentile}\PY{p}{(}\PY{n}{grand\PYZus{}sample\PYZus{}means}\PY{p}{,} \PY{l+m+mf}{97.5}\PY{p}{)}
        
        \PY{k}{return} \PY{p}{\PYZob{}}
            \PY{l+s+s1}{\PYZsq{}}\PY{l+s+s1}{first\PYZus{}CI}\PY{l+s+s1}{\PYZsq{}} \PY{p}{:} \PY{p}{[}\PY{n}{first\PYZus{}lb}\PY{p}{,} \PY{n}{first\PYZus{}rb}\PY{p}{]}\PY{p}{,}
            \PY{l+s+s1}{\PYZsq{}}\PY{l+s+s1}{second\PYZus{}CI}\PY{l+s+s1}{\PYZsq{}} \PY{p}{:} \PY{p}{[}\PY{n}{second\PYZus{}lb}\PY{p}{,} \PY{n}{second\PYZus{}rb}\PY{p}{]}\PY{p}{,}
            \PY{l+s+s1}{\PYZsq{}}\PY{l+s+s1}{grand\PYZus{}CI}\PY{l+s+s1}{\PYZsq{}} \PY{p}{:} \PY{p}{[}\PY{n}{grand\PYZus{}lb}\PY{p}{,} \PY{n}{grand\PYZus{}rb}\PY{p}{]}
        \PY{p}{\PYZcb{}}
\end{Verbatim}
\end{tcolorbox}

    Pivotal version:

    \begin{tcolorbox}[breakable, size=fbox, boxrule=1pt, pad at break*=1mm,colback=cellbackground, colframe=cellborder]
\prompt{In}{incolor}{21}{\boxspacing}
\begin{Verbatim}[commandchars=\\\{\}]
\PY{n}{bootstrap}\PY{p}{(}\PY{n}{train}\PY{p}{,} \PY{l+s+s1}{\PYZsq{}}\PY{l+s+s1}{count}\PY{l+s+s1}{\PYZsq{}}\PY{p}{,} \PY{l+s+s1}{\PYZsq{}}\PY{l+s+s1}{clusters\PYZus{}five}\PY{l+s+s1}{\PYZsq{}}\PY{p}{,} \PY{l+m+mi}{0}\PY{p}{,} \PY{l+m+mi}{3}\PY{p}{)}
\end{Verbatim}
\end{tcolorbox}

    \begin{Verbatim}[commandchars=\\\{\}]
191.64603999999997
    \end{Verbatim}

            \begin{tcolorbox}[breakable, size=fbox, boxrule=.5pt, pad at break*=1mm, opacityfill=0]
\prompt{Out}{outcolor}{21}{\boxspacing}
\begin{Verbatim}[commandchars=\\\{\}]
\{'first\_CI': [189.56506920613344, 193.7270107938665],
 'second\_CI': [231.78550413444978, 240.08575586555023],
 'grand\_CI': [146.39047234839848, 151.7739476516016]\}
\end{Verbatim}
\end{tcolorbox}
        
    Non-pivotal version:

    \begin{tcolorbox}[breakable, size=fbox, boxrule=1pt, pad at break*=1mm,colback=cellbackground, colframe=cellborder]
\prompt{In}{incolor}{22}{\boxspacing}
\begin{Verbatim}[commandchars=\\\{\}]
\PY{n}{bootstrap}\PY{p}{(}\PY{n}{train}\PY{p}{,} \PY{l+s+s1}{\PYZsq{}}\PY{l+s+s1}{count}\PY{l+s+s1}{\PYZsq{}}\PY{p}{,} \PY{l+s+s1}{\PYZsq{}}\PY{l+s+s1}{clusters\PYZus{}five}\PY{l+s+s1}{\PYZsq{}}\PY{p}{,} \PY{l+m+mi}{0}\PY{p}{,} \PY{l+m+mi}{3}\PY{p}{,} \PY{k+kc}{False}\PY{p}{)}
\end{Verbatim}
\end{tcolorbox}

            \begin{tcolorbox}[breakable, size=fbox, boxrule=.5pt, pad at break*=1mm, opacityfill=0]
\prompt{Out}{outcolor}{22}{\boxspacing}
\begin{Verbatim}[commandchars=\\\{\}]
\{'first\_CI': [189.59925, 193.88025],
 'second\_CI': [231.93975, 240.45125],
 'grand\_CI': [146.32350000000002, 151.62405]\}
\end{Verbatim}
\end{tcolorbox}
        
    Conclusion:

\begin{itemize}
\item
  We could see that pivotal and non-pivotal bootstrap has almost no
  difference in results and speed. It means we could use both of the
  methods for our calculation
\item
  If we compare two clusters we will see that first one has a lot of
  ones and that's why CI for it is much different. It is caused by the
  number of ones that appeared in the cluster zero.
\item
  Bootstrap estimate for the second CI and grand CI are almost the same.
  If we look at the distribution of classes in cluster in\textbar{} the
  previous task we will notice that the distributions are very similar.
\end{itemize}

\subsection*{Contingency Table}

    Let us take the following features for the analysis:

\begin{itemize}
\item
  count
\item
  vert\_symmetry
\item
  hor\_mass\_center
\end{itemize}

Now we will look at their histogramms and choose the cutoffs:

    \begin{tcolorbox}[breakable, size=fbox, boxrule=1pt, pad at break*=1mm,colback=cellbackground, colframe=cellborder]
\prompt{In}{incolor}{23}{\boxspacing}
\begin{Verbatim}[commandchars=\\\{\}]
\PY{n}{plt}\PY{o}{.}\PY{n}{figure}\PY{p}{(}\PY{p}{)}
\PY{n}{f}\PY{p}{,} \PY{p}{(}\PY{n}{ax1}\PY{p}{,} \PY{n}{ax2}\PY{p}{,} \PY{n}{ax3}\PY{p}{)} \PY{o}{=} \PY{n}{plt}\PY{o}{.}\PY{n}{subplots}\PY{p}{(}\PY{l+m+mi}{1}\PY{p}{,} \PY{l+m+mi}{3}\PY{p}{,} \PY{n}{sharey}\PY{o}{=}\PY{k+kc}{True}\PY{p}{,} \PY{n}{figsize}\PY{o}{=}\PY{p}{(}\PY{l+m+mi}{20}\PY{p}{,} \PY{l+m+mi}{10}\PY{p}{)}\PY{p}{)}
\PY{n}{ax1}\PY{o}{.}\PY{n}{hist}\PY{p}{(}\PY{n}{train}\PY{p}{[}\PY{l+s+s1}{\PYZsq{}}\PY{l+s+s1}{count}\PY{l+s+s1}{\PYZsq{}}\PY{p}{]}\PY{p}{,} \PY{n}{bins} \PY{o}{=} \PY{l+m+mi}{30}\PY{p}{)}
\PY{n}{ax1}\PY{o}{.}\PY{n}{set\PYZus{}title}\PY{p}{(}\PY{l+s+s1}{\PYZsq{}}\PY{l+s+s1}{Count feature hist}\PY{l+s+s1}{\PYZsq{}}\PY{p}{)}
\PY{n}{ax2}\PY{o}{.}\PY{n}{hist}\PY{p}{(}\PY{n}{train}\PY{p}{[}\PY{l+s+s1}{\PYZsq{}}\PY{l+s+s1}{vert\PYZus{}symmetry}\PY{l+s+s1}{\PYZsq{}}\PY{p}{]}\PY{p}{,} \PY{n}{bins} \PY{o}{=} \PY{l+m+mi}{30}\PY{p}{)}
\PY{n}{ax2}\PY{o}{.}\PY{n}{set\PYZus{}title}\PY{p}{(}\PY{l+s+s1}{\PYZsq{}}\PY{l+s+s1}{Vert symmetry feature hist}\PY{l+s+s1}{\PYZsq{}}\PY{p}{)}
\PY{n}{ax3}\PY{o}{.}\PY{n}{hist}\PY{p}{(}\PY{n}{train}\PY{p}{[}\PY{l+s+s1}{\PYZsq{}}\PY{l+s+s1}{hor\PYZus{}mass\PYZus{}center}\PY{l+s+s1}{\PYZsq{}}\PY{p}{]}\PY{p}{,} \PY{n}{bins} \PY{o}{=} \PY{l+m+mi}{30}\PY{p}{)}
\PY{n}{ax3}\PY{o}{.}\PY{n}{set\PYZus{}title}\PY{p}{(}\PY{l+s+s1}{\PYZsq{}}\PY{l+s+s1}{Hor mass center feature hist}\PY{l+s+s1}{\PYZsq{}}\PY{p}{)}
\end{Verbatim}
\end{tcolorbox}

            \begin{tcolorbox}[breakable, size=fbox, boxrule=.5pt, pad at break*=1mm, opacityfill=0]
\prompt{Out}{outcolor}{23}{\boxspacing}
\begin{Verbatim}[commandchars=\\\{\}]
Text(0.5, 1.0, 'Hor mass center feature hist')
\end{Verbatim}
\end{tcolorbox}
        
    
    \begin{verbatim}
<Figure size 432x288 with 0 Axes>
    \end{verbatim}

    
    \begin{center}
    \adjustimage{max size={0.9\linewidth}{0.9\paperheight}}{other/output_32_2.png}
    \end{center}
    { \hspace*{\fill} \\}
    
    Boundaries for the first feature:

\begin{itemize}
\item
  0-100
\item
  100-175
\item
  175-300
\item
  300-400
\end{itemize}

Boundaries for the second feature:

\begin{itemize}
\item
  -35-(-10)
\item
  -10-0
\item
  0-20
\item
  20-40
\end{itemize}

Boundaries for the third feature:

\begin{itemize}
\item
  0-13.7
\item
  13.7-14.1
\item
  14.2-\ldots{}
\end{itemize}

Now we will use method from pandas library for categorizing the feature.

    \begin{tcolorbox}[breakable, size=fbox, boxrule=1pt, pad at break*=1mm,colback=cellbackground, colframe=cellborder]
\prompt{In}{incolor}{24}{\boxspacing}
\begin{Verbatim}[commandchars=\\\{\}]
\PY{n}{train}\PY{p}{[}\PY{l+s+s1}{\PYZsq{}}\PY{l+s+s1}{count\PYZus{}cat}\PY{l+s+s1}{\PYZsq{}}\PY{p}{]} \PY{o}{=} \PY{n}{pd}\PY{o}{.}\PY{n}{cut}\PY{p}{(}\PY{n}{train}\PY{p}{[}\PY{l+s+s1}{\PYZsq{}}\PY{l+s+s1}{count}\PY{l+s+s1}{\PYZsq{}}\PY{p}{]}\PY{p}{,}
                     \PY{n}{bins}\PY{o}{=}\PY{p}{[}\PY{l+m+mi}{0}\PY{p}{,} \PY{l+m+mi}{100}\PY{p}{,} \PY{l+m+mi}{175}\PY{p}{,} \PY{l+m+mi}{300}\PY{p}{,} \PY{l+m+mi}{400}\PY{p}{]}\PY{p}{,}
                     \PY{n}{labels}\PY{o}{=}\PY{p}{[}\PY{l+s+s2}{\PYZdq{}}\PY{l+s+s2}{0\PYZhy{}100}\PY{l+s+s2}{\PYZdq{}}\PY{p}{,} \PY{l+s+s2}{\PYZdq{}}\PY{l+s+s2}{100\PYZhy{}175}\PY{l+s+s2}{\PYZdq{}}\PY{p}{,} \PY{l+s+s2}{\PYZdq{}}\PY{l+s+s2}{175\PYZhy{}300}\PY{l+s+s2}{\PYZdq{}}\PY{p}{,} \PY{l+s+s2}{\PYZdq{}}\PY{l+s+s2}{300\PYZhy{}400}\PY{l+s+s2}{\PYZdq{}}\PY{p}{]}\PY{p}{)}
\end{Verbatim}
\end{tcolorbox}

    \begin{tcolorbox}[breakable, size=fbox, boxrule=1pt, pad at break*=1mm,colback=cellbackground, colframe=cellborder]
\prompt{In}{incolor}{25}{\boxspacing}
\begin{Verbatim}[commandchars=\\\{\}]
\PY{n}{train}\PY{p}{[}\PY{l+s+s1}{\PYZsq{}}\PY{l+s+s1}{vert\PYZus{}symmetry\PYZus{}cat}\PY{l+s+s1}{\PYZsq{}}\PY{p}{]} \PY{o}{=} \PY{n}{pd}\PY{o}{.}\PY{n}{cut}\PY{p}{(}\PY{n}{train}\PY{p}{[}\PY{l+s+s1}{\PYZsq{}}\PY{l+s+s1}{vert\PYZus{}symmetry}\PY{l+s+s1}{\PYZsq{}}\PY{p}{]}\PY{p}{,}
                     \PY{n}{bins}\PY{o}{=}\PY{p}{[}\PY{o}{\PYZhy{}}\PY{l+m+mi}{35}\PY{p}{,} \PY{o}{\PYZhy{}}\PY{l+m+mi}{10}\PY{p}{,} \PY{l+m+mi}{0}\PY{p}{,} \PY{l+m+mi}{20}\PY{p}{,} \PY{l+m+mi}{40}\PY{p}{]}\PY{p}{,}
                     \PY{n}{labels}\PY{o}{=}\PY{p}{[}\PY{l+s+s2}{\PYZdq{}}\PY{l+s+s2}{\PYZhy{}35\PYZhy{}(\PYZhy{}10)}\PY{l+s+s2}{\PYZdq{}}\PY{p}{,} \PY{l+s+s2}{\PYZdq{}}\PY{l+s+s2}{\PYZhy{}10\PYZhy{}0}\PY{l+s+s2}{\PYZdq{}}\PY{p}{,} \PY{l+s+s2}{\PYZdq{}}\PY{l+s+s2}{0\PYZhy{}20}\PY{l+s+s2}{\PYZdq{}}\PY{p}{,} \PY{l+s+s2}{\PYZdq{}}\PY{l+s+s2}{20\PYZhy{}40}\PY{l+s+s2}{\PYZdq{}}\PY{p}{]}\PY{p}{)}
\end{Verbatim}
\end{tcolorbox}

    \begin{tcolorbox}[breakable, size=fbox, boxrule=1pt, pad at break*=1mm,colback=cellbackground, colframe=cellborder]
\prompt{In}{incolor}{26}{\boxspacing}
\begin{Verbatim}[commandchars=\\\{\}]
\PY{n}{train}\PY{p}{[}\PY{l+s+s1}{\PYZsq{}}\PY{l+s+s1}{hor\PYZus{}mass\PYZus{}center\PYZus{}cat}\PY{l+s+s1}{\PYZsq{}}\PY{p}{]} \PY{o}{=} \PY{n}{pd}\PY{o}{.}\PY{n}{cut}\PY{p}{(}\PY{n}{train}\PY{p}{[}\PY{l+s+s1}{\PYZsq{}}\PY{l+s+s1}{hor\PYZus{}mass\PYZus{}center}\PY{l+s+s1}{\PYZsq{}}\PY{p}{]}\PY{p}{,}
                     \PY{n}{bins}\PY{o}{=}\PY{p}{[}\PY{l+m+mi}{0}\PY{p}{,} \PY{l+m+mf}{13.7}\PY{p}{,} \PY{l+m+mf}{14.1}\PY{p}{,} \PY{l+m+mf}{14.5}\PY{p}{]}\PY{p}{,}
                     \PY{n}{labels}\PY{o}{=}\PY{p}{[}\PY{l+s+s2}{\PYZdq{}}\PY{l+s+s2}{0\PYZhy{}13.7}\PY{l+s+s2}{\PYZdq{}}\PY{p}{,} \PY{l+s+s2}{\PYZdq{}}\PY{l+s+s2}{13.7\PYZhy{}14.1}\PY{l+s+s2}{\PYZdq{}}\PY{p}{,} \PY{l+s+s2}{\PYZdq{}}\PY{l+s+s2}{14.2\PYZhy{}...}\PY{l+s+s2}{\PYZdq{}}\PY{p}{]}\PY{p}{)}
\end{Verbatim}
\end{tcolorbox}

    Now we will create conditional frequency table, based on our categorized
features:

    \begin{tcolorbox}[breakable, size=fbox, boxrule=1pt, pad at break*=1mm,colback=cellbackground, colframe=cellborder]
\prompt{In}{incolor}{27}{\boxspacing}
\begin{Verbatim}[commandchars=\\\{\}]
\PY{n}{pd}\PY{o}{.}\PY{n}{crosstab}\PY{p}{(}\PY{n}{train}\PY{p}{[}\PY{l+s+s1}{\PYZsq{}}\PY{l+s+s1}{label}\PY{l+s+s1}{\PYZsq{}}\PY{p}{]}\PY{p}{,} \PY{n}{train}\PY{p}{[}\PY{l+s+s1}{\PYZsq{}}\PY{l+s+s1}{count\PYZus{}cat}\PY{l+s+s1}{\PYZsq{}}\PY{p}{]}\PY{p}{,} 
            \PY{n}{margins}\PY{o}{=}\PY{k+kc}{True}\PY{p}{,} \PY{n}{margins\PYZus{}name}\PY{o}{=}\PY{l+s+s1}{\PYZsq{}}\PY{l+s+s1}{Total}\PY{l+s+s1}{\PYZsq{}}\PY{p}{)}
\end{Verbatim}
\end{tcolorbox}

            \begin{tcolorbox}[breakable, size=fbox, boxrule=.5pt, pad at break*=1mm, opacityfill=0]
\prompt{Out}{outcolor}{27}{\boxspacing}
\begin{Verbatim}[commandchars=\\\{\}]
count\_cat  0-100  100-175  175-300  300-400  Total
label
0              0       33       77        1    111
1             88       22        0        0    110
2              1       64       34        0     99
3              4       48       31        0     83
4              2       84        9        0     95
5              4       77       12        0     93
6              5       69       16        0     90
7             19       82        5        0    106
8              0       56       57        0    113
9              3       84       13        0    100
Total        126      619      254        1   1000
\end{Verbatim}
\end{tcolorbox}
        
    \begin{tcolorbox}[breakable, size=fbox, boxrule=1pt, pad at break*=1mm,colback=cellbackground, colframe=cellborder]
\prompt{In}{incolor}{28}{\boxspacing}
\begin{Verbatim}[commandchars=\\\{\}]
\PY{n}{pd}\PY{o}{.}\PY{n}{crosstab}\PY{p}{(}\PY{n}{train}\PY{p}{[}\PY{l+s+s1}{\PYZsq{}}\PY{l+s+s1}{label}\PY{l+s+s1}{\PYZsq{}}\PY{p}{]}\PY{p}{,} \PY{n}{train}\PY{p}{[}\PY{l+s+s1}{\PYZsq{}}\PY{l+s+s1}{vert\PYZus{}symmetry\PYZus{}cat}\PY{l+s+s1}{\PYZsq{}}\PY{p}{]}\PY{p}{,} 
            \PY{n}{margins}\PY{o}{=}\PY{k+kc}{True}\PY{p}{,} \PY{n}{margins\PYZus{}name}\PY{o}{=}\PY{l+s+s1}{\PYZsq{}}\PY{l+s+s1}{Total}\PY{l+s+s1}{\PYZsq{}}\PY{p}{)}
\end{Verbatim}
\end{tcolorbox}

            \begin{tcolorbox}[breakable, size=fbox, boxrule=.5pt, pad at break*=1mm, opacityfill=0]
\prompt{Out}{outcolor}{28}{\boxspacing}
\begin{Verbatim}[commandchars=\\\{\}]
vert\_symmetry\_cat  -35-(-10)  -10-0  0-20  20-40  Total
label
0                          8     88    15      0    111
1                          1    104     5      0    110
2                         68     27     4      0     99
3                          4     53    26      0     83
4                         32     52    11      0     95
5                          8     44    41      0     93
6                         66     23     1      0     90
7                          2     16    87      1    106
8                          7     70    36      0    113
9                          2     57    41      0    100
Total                    198    534   267      1   1000
\end{Verbatim}
\end{tcolorbox}
        
    \begin{tcolorbox}[breakable, size=fbox, boxrule=1pt, pad at break*=1mm,colback=cellbackground, colframe=cellborder]
\prompt{In}{incolor}{29}{\boxspacing}
\begin{Verbatim}[commandchars=\\\{\}]
\PY{n}{pd}\PY{o}{.}\PY{n}{crosstab}\PY{p}{(}\PY{n}{train}\PY{p}{[}\PY{l+s+s1}{\PYZsq{}}\PY{l+s+s1}{label}\PY{l+s+s1}{\PYZsq{}}\PY{p}{]}\PY{p}{,} \PY{n}{train}\PY{p}{[}\PY{l+s+s1}{\PYZsq{}}\PY{l+s+s1}{hor\PYZus{}mass\PYZus{}center\PYZus{}cat}\PY{l+s+s1}{\PYZsq{}}\PY{p}{]}\PY{p}{,} 
            \PY{n}{margins}\PY{o}{=}\PY{k+kc}{True}\PY{p}{,} \PY{n}{margins\PYZus{}name}\PY{o}{=}\PY{l+s+s1}{\PYZsq{}}\PY{l+s+s1}{Total}\PY{l+s+s1}{\PYZsq{}}\PY{p}{)}
\end{Verbatim}
\end{tcolorbox}

            \begin{tcolorbox}[breakable, size=fbox, boxrule=.5pt, pad at break*=1mm, opacityfill=0]
\prompt{Out}{outcolor}{29}{\boxspacing}
\begin{Verbatim}[commandchars=\\\{\}]
hor\_mass\_center\_cat  0-13.7  13.7-14.1  14.2-{\ldots}  Total
label
0                        20         46        45    111
1                        25         36        49    110
2                        21         44        34     99
3                        13         24        46     83
4                        15         34        46     95
5                        20         43        30     93
6                        14         37        39     90
7                        22         37        47    106
8                        20         46        47    113
9                        20         31        49    100
Total                   190        378       432   1000
\end{Verbatim}
\end{tcolorbox}
        
    From this table we could see the following:

\begin{itemize}
\item
  The most frequent result if there less than 100 pixels is number 1,
  because we need only few pixels to draw one.
\item
  If there are more than 300 pixels it means that this an outlier for
  the dataset because there is only one such sample
\item
  When talking about vertical simmetry we could notice that the in the
  range 0-20 the most frequent digit is seven. This is normal because
  this digit has more pixel in the upper part.
\item
  This is also true for the ones. But for most of them simmetry is near
  zero beacause of its form :)
\item
  Hor mass center feature has distribution form closed to the Uniform
  that is what we nearly see in the frequency table
\end{itemize}

Now, for calculating quetlet table we define a function that takes the
following arguments:

\begin{itemize}
\item
  X - data
\item
  first\_feature - first feature for calculating Quételet index
\item
  second\_feature - second feature for calculating Quételet index
\item
  :return: dict with table and summary Quetelet index
\end{itemize}

This function iterates through all possible values of the both features
and then calculates the index by deviding the probabilities. Also, it
calculates chi square using the same probs.

    \begin{tcolorbox}[breakable, size=fbox, boxrule=1pt, pad at break*=1mm,colback=cellbackground, colframe=cellborder]
\prompt{In}{incolor}{30}{\boxspacing}
\begin{Verbatim}[commandchars=\\\{\}]
\PY{k}{def} \PY{n+nf}{build\PYZus{}quetlet\PYZus{}table}\PY{p}{(}\PY{n}{X}\PY{p}{,} \PY{n}{first\PYZus{}feature}\PY{p}{,} \PY{n}{second\PYZus{}feature}\PY{p}{)}\PY{p}{:}
    \PY{n}{quetlet\PYZus{}table} \PY{o}{=} \PY{p}{\PYZob{}}\PY{p}{\PYZcb{}}
    \PY{n}{chi\PYZus{}square} \PY{o}{=} \PY{l+m+mi}{0}
    \PY{k}{for} \PY{n}{k} \PY{o+ow}{in} \PY{n}{X}\PY{p}{[}\PY{n}{first\PYZus{}feature}\PY{p}{]}\PY{o}{.}\PY{n}{unique}\PY{p}{(}\PY{p}{)}\PY{p}{:}
        \PY{n}{quetlet\PYZus{}index} \PY{o}{=} \PY{n+nb}{dict}\PY{p}{(}\PY{p}{)}
        \PY{k}{for} \PY{n}{l} \PY{o+ow}{in} \PY{n}{X}\PY{p}{[}\PY{n}{second\PYZus{}feature}\PY{p}{]}\PY{o}{.}\PY{n}{unique}\PY{p}{(}\PY{p}{)}\PY{p}{:}
            \PY{n}{first\PYZus{}count} \PY{o}{=} \PY{p}{(}\PY{n}{X}\PY{p}{[}\PY{n}{first\PYZus{}feature}\PY{p}{]} \PY{o}{==} \PY{n}{k}\PY{p}{)}
            \PY{n}{second\PYZus{}count} \PY{o}{=} \PY{p}{(}\PY{n}{X}\PY{p}{[}\PY{n}{second\PYZus{}feature}\PY{p}{]} \PY{o}{==} \PY{n}{l}\PY{p}{)}
            \PY{n}{p\PYZus{}hg} \PY{o}{=} \PY{p}{(}\PY{n}{first\PYZus{}count} \PY{o}{\PYZam{}} \PY{n}{second\PYZus{}count}\PY{p}{)}\PY{o}{.}\PY{n}{mean}\PY{p}{(}\PY{p}{)}
            \PY{n}{p\PYZus{}h} \PY{o}{=} \PY{n}{first\PYZus{}count}\PY{o}{.}\PY{n}{mean}\PY{p}{(}\PY{p}{)}
            \PY{n}{p\PYZus{}g} \PY{o}{=} \PY{n}{second\PYZus{}count}\PY{o}{.}\PY{n}{mean}\PY{p}{(}\PY{p}{)}
            \PY{n}{quetlet\PYZus{}index}\PY{p}{[}\PY{n}{l}\PY{p}{]} \PY{o}{=} \PY{n}{p\PYZus{}hg} \PY{o}{/} \PY{p}{(}\PY{n}{p\PYZus{}h} \PY{o}{*} \PY{n}{p\PYZus{}g}\PY{p}{)} \PY{o}{\PYZhy{}} \PY{l+m+mi}{1}
            \PY{n}{chi\PYZus{}square} \PY{o}{+}\PY{o}{=} \PY{p}{(}\PY{n}{p\PYZus{}hg} \PY{o}{\PYZhy{}} \PY{n}{p\PYZus{}h} \PY{o}{*} \PY{n}{p\PYZus{}g}\PY{p}{)}\PY{o}{*}\PY{o}{*}\PY{l+m+mi}{2} \PY{o}{/} \PY{p}{(}\PY{n}{p\PYZus{}h} \PY{o}{*} \PY{n}{p\PYZus{}g} \PY{o}{+} \PY{l+m+mf}{1e\PYZhy{}10}\PY{p}{)}
        \PY{n}{quetlet\PYZus{}table}\PY{p}{[}\PY{n}{k}\PY{p}{]} \PY{o}{=} \PY{n}{quetlet\PYZus{}index}
    \PY{k}{return} \PY{n}{quetlet\PYZus{}table}\PY{p}{,} \PY{n}{chi\PYZus{}square}
\end{Verbatim}
\end{tcolorbox}

    \begin{tcolorbox}[breakable, size=fbox, boxrule=1pt, pad at break*=1mm,colback=cellbackground, colframe=cellborder]
\prompt{In}{incolor}{31}{\boxspacing}
\begin{Verbatim}[commandchars=\\\{\}]
\PY{n}{table\PYZus{}count\PYZus{}hor}\PY{p}{,} \PY{n}{chi\PYZus{}count\PYZus{}hor} \PY{o}{=} \PY{n}{build\PYZus{}quetlet\PYZus{}table}\PY{p}{(}\PY{n}{train}\PY{p}{,} \PY{l+s+s1}{\PYZsq{}}\PY{l+s+s1}{count\PYZus{}cat}\PY{l+s+s1}{\PYZsq{}}\PY{p}{,} \PY{l+s+s1}{\PYZsq{}}\PY{l+s+s1}{hor\PYZus{}mass\PYZus{}center\PYZus{}cat}\PY{l+s+s1}{\PYZsq{}}\PY{p}{)}
\end{Verbatim}
\end{tcolorbox}

    \begin{tcolorbox}[breakable, size=fbox, boxrule=1pt, pad at break*=1mm,colback=cellbackground, colframe=cellborder]
\prompt{In}{incolor}{32}{\boxspacing}
\begin{Verbatim}[commandchars=\\\{\}]
\PY{n}{chi\PYZus{}count\PYZus{}hor}
\end{Verbatim}
\end{tcolorbox}

            \begin{tcolorbox}[breakable, size=fbox, boxrule=.5pt, pad at break*=1mm, opacityfill=0]
\prompt{Out}{outcolor}{32}{\boxspacing}
\begin{Verbatim}[commandchars=\\\{\}]
0.006041453360232884
\end{Verbatim}
\end{tcolorbox}
        
    \begin{tcolorbox}[breakable, size=fbox, boxrule=1pt, pad at break*=1mm,colback=cellbackground, colframe=cellborder]
\prompt{In}{incolor}{33}{\boxspacing}
\begin{Verbatim}[commandchars=\\\{\}]
\PY{n}{table\PYZus{}count\PYZus{}vert}\PY{p}{,} \PY{n}{chi\PYZus{}count\PYZus{}vert} \PY{o}{=}  \PY{n}{build\PYZus{}quetlet\PYZus{}table}\PY{p}{(}\PY{n}{train}\PY{p}{,} \PY{l+s+s1}{\PYZsq{}}\PY{l+s+s1}{count\PYZus{}cat}\PY{l+s+s1}{\PYZsq{}}\PY{p}{,} \PY{l+s+s1}{\PYZsq{}}\PY{l+s+s1}{vert\PYZus{}symmetry\PYZus{}cat}\PY{l+s+s1}{\PYZsq{}}\PY{p}{)}
\end{Verbatim}
\end{tcolorbox}

    \begin{tcolorbox}[breakable, size=fbox, boxrule=1pt, pad at break*=1mm,colback=cellbackground, colframe=cellborder]
\prompt{In}{incolor}{34}{\boxspacing}
\begin{Verbatim}[commandchars=\\\{\}]
\PY{n}{chi\PYZus{}count\PYZus{}vert}
\end{Verbatim}
\end{tcolorbox}

            \begin{tcolorbox}[breakable, size=fbox, boxrule=.5pt, pad at break*=1mm, opacityfill=0]
\prompt{Out}{outcolor}{34}{\boxspacing}
\begin{Verbatim}[commandchars=\\\{\}]
0.06065758855161415
\end{Verbatim}
\end{tcolorbox}
        
    \begin{tcolorbox}[breakable, size=fbox, boxrule=1pt, pad at break*=1mm,colback=cellbackground, colframe=cellborder]
\prompt{In}{incolor}{35}{\boxspacing}
\begin{Verbatim}[commandchars=\\\{\}]
\PY{n}{pd}\PY{o}{.}\PY{n}{DataFrame}\PY{p}{(}\PY{n}{table\PYZus{}count\PYZus{}vert}\PY{p}{)}
\end{Verbatim}
\end{tcolorbox}

            \begin{tcolorbox}[breakable, size=fbox, boxrule=.5pt, pad at break*=1mm, opacityfill=0]
\prompt{Out}{outcolor}{35}{\boxspacing}
\begin{Verbatim}[commandchars=\\\{\}]
            100-175   175-300     0-100   300-400
0-20       0.173815 -0.233242 -0.375780 -1.000000
-10-0     -0.116613  0.024802  0.515962  0.872659
-35-(-10)  0.077006  0.252684 -0.879750 -1.000000
20-40      0.615509 -1.000000 -1.000000 -1.000000
\end{Verbatim}
\end{tcolorbox}
        
    \begin{tcolorbox}[breakable, size=fbox, boxrule=1pt, pad at break*=1mm,colback=cellbackground, colframe=cellborder]
\prompt{In}{incolor}{36}{\boxspacing}
\begin{Verbatim}[commandchars=\\\{\}]
\PY{n}{pd}\PY{o}{.}\PY{n}{DataFrame}\PY{p}{(}\PY{n}{table\PYZus{}count\PYZus{}hor}\PY{p}{)}
\end{Verbatim}
\end{tcolorbox}

            \begin{tcolorbox}[breakable, size=fbox, boxrule=.5pt, pad at break*=1mm, opacityfill=0]
\prompt{Out}{outcolor}{36}{\boxspacing}
\begin{Verbatim}[commandchars=\\\{\}]
            100-175   175-300     0-100   300-400
14.2-{\ldots}  -0.031443  0.057160  0.028807  1.314815
13.7-14.1  0.064184 -0.083448 -0.139162 -1.000000
0-13.7    -0.056203  0.036055  0.211362 -1.000000
\end{Verbatim}
\end{tcolorbox}
        
    Conclusions:

\begin{itemize}
\item
  From both summary Quetelet index we could see that this features are
  most likely to be independent.
\item
  We see that value 0-100 is 51\% more likey to appear when
  vert\_symmetry has value -10-0
\item
  Also we see that value 300-400 is 87\% more likey to appear when
  vert\_symmetry has value -10-0
\end{itemize}

    Let's calculate the number of observations that would suffice to see the
features as associated. Remember that if the hypothesis of independence
is true, then:

\(NX^2 \sim \chi^2((K - 1)(L-1))\), where \(K, L\) are numbers of
possible different values for the features.

In this case, we have 2 degrees of freedom:

    \begin{tcolorbox}[breakable, size=fbox, boxrule=1pt, pad at break*=1mm,colback=cellbackground, colframe=cellborder]
\prompt{In}{incolor}{40}{\boxspacing}
\begin{Verbatim}[commandchars=\\\{\}]
\PY{n+nb}{print}\PY{p}{(}\PY{n}{chi2}\PY{p}{(}\PY{n}{df}\PY{o}{=}\PY{l+m+mi}{2}\PY{p}{)}\PY{o}{.}\PY{n}{ppf}\PY{p}{(}\PY{l+m+mf}{0.95}\PY{p}{)}\PY{p}{)}
\PY{n+nb}{print}\PY{p}{(}\PY{n}{chi2}\PY{p}{(}\PY{n}{df}\PY{o}{=}\PY{l+m+mi}{2}\PY{p}{)}\PY{o}{.}\PY{n}{ppf}\PY{p}{(}\PY{l+m+mf}{0.99}\PY{p}{)}\PY{p}{)}
\end{Verbatim}
\end{tcolorbox}

    \begin{Verbatim}[commandchars=\\\{\}]
5.991464547107979
9.21034037197618
    \end{Verbatim}

    If the features are independent, there is only \(5\%\) chance that
\(NX^2\) will be greater than \(5.99\), and \(1\%\) chance that \(NX^2\)
will be greater than \(9.21\). We want to find such \(N\) that \(NX^2\)
will exceed specified values, so we will calculate \(5.99\) and
\(\frac{9.21}{X^2}\)

    \begin{tcolorbox}[breakable, size=fbox, boxrule=1pt, pad at break*=1mm,colback=cellbackground, colframe=cellborder]
\prompt{In}{incolor}{41}{\boxspacing}
\begin{Verbatim}[commandchars=\\\{\}]
\PY{k}{def} \PY{n+nf}{chi\PYZus{}square}\PY{p}{(}\PY{n}{dataframe}\PY{p}{,} \PY{n}{ver\PYZus{}feature}\PY{p}{,} \PY{n}{hor\PYZus{}feature}\PY{p}{)}\PY{p}{:}
    \PY{n}{ver\PYZus{}column} \PY{o}{=} \PY{n}{dataframe}\PY{p}{[}\PY{n}{ver\PYZus{}feature}\PY{p}{]}
    \PY{n}{hor\PYZus{}column} \PY{o}{=} \PY{n}{dataframe}\PY{p}{[}\PY{n}{hor\PYZus{}feature}\PY{p}{]}
    \PY{n}{result} \PY{o}{=} \PY{l+m+mi}{0}
    \PY{k}{for} \PY{n}{ver\PYZus{}value} \PY{o+ow}{in} \PY{n+nb}{sorted}\PY{p}{(}\PY{n}{ver\PYZus{}column}\PY{o}{.}\PY{n}{unique}\PY{p}{(}\PY{p}{)}\PY{p}{)}\PY{p}{:}
        \PY{k}{for} \PY{n}{hor\PYZus{}value} \PY{o+ow}{in} \PY{n+nb}{sorted}\PY{p}{(}\PY{n}{hor\PYZus{}column}\PY{o}{.}\PY{n}{unique}\PY{p}{(}\PY{p}{)}\PY{p}{)}\PY{p}{:}
            \PY{n}{p\PYZus{}vk} \PY{o}{=} \PY{p}{(}\PY{p}{(}\PY{n}{ver\PYZus{}column} \PY{o}{==} \PY{n}{ver\PYZus{}value}\PY{p}{)} \PY{o}{\PYZam{}}
                    \PY{p}{(}\PY{n}{hor\PYZus{}column} \PY{o}{==} \PY{n}{hor\PYZus{}value}\PY{p}{)}\PY{p}{)}\PY{o}{.}\PY{n}{mean}\PY{p}{(}\PY{p}{)}
            \PY{n}{p\PYZus{}k} \PY{o}{=} \PY{p}{(}\PY{n}{ver\PYZus{}column} \PY{o}{==} \PY{n}{ver\PYZus{}value}\PY{p}{)}\PY{o}{.}\PY{n}{mean}\PY{p}{(}\PY{p}{)}
            \PY{n}{p\PYZus{}v} \PY{o}{=} \PY{p}{(}\PY{n}{hor\PYZus{}column} \PY{o}{==} \PY{n}{hor\PYZus{}value}\PY{p}{)}\PY{o}{.}\PY{n}{mean}\PY{p}{(}\PY{p}{)}
            \PY{n}{result} \PY{o}{+}\PY{o}{=} \PY{p}{(}\PY{n}{p\PYZus{}vk} \PY{o}{\PYZhy{}} \PY{n}{p\PYZus{}k} \PY{o}{*} \PY{n}{p\PYZus{}v}\PY{p}{)}\PY{o}{*}\PY{o}{*}\PY{l+m+mi}{2} \PY{o}{/} \PY{p}{(}\PY{n}{p\PYZus{}k} \PY{o}{*} \PY{n}{p\PYZus{}v}\PY{p}{)}
    \PY{n}{max\PYZus{}result} \PY{o}{=} \PY{n+nb}{min}\PY{p}{(}\PY{n+nb}{len}\PY{p}{(}\PY{n}{ver\PYZus{}column}\PY{o}{.}\PY{n}{unique}\PY{p}{(}\PY{p}{)}\PY{p}{)}\PY{p}{,} \PY{n+nb}{len}\PY{p}{(}\PY{n}{hor\PYZus{}column}\PY{o}{.}\PY{n}{unique}\PY{p}{(}\PY{p}{)}\PY{p}{)}\PY{p}{)} \PY{o}{\PYZhy{}} \PY{l+m+mi}{1}
    \PY{k}{return} \PY{n}{result}\PY{p}{,} \PY{n}{max\PYZus{}result}
\end{Verbatim}
\end{tcolorbox}

    \begin{tcolorbox}[breakable, size=fbox, boxrule=1pt, pad at break*=1mm,colback=cellbackground, colframe=cellborder]
\prompt{In}{incolor}{43}{\boxspacing}
\begin{Verbatim}[commandchars=\\\{\}]
\PY{n}{feature1} \PY{o}{=} \PY{l+s+s1}{\PYZsq{}}\PY{l+s+s1}{count\PYZus{}cat}\PY{l+s+s1}{\PYZsq{}}
\PY{k}{for} \PY{n}{feature2} \PY{o+ow}{in} \PY{p}{(}\PY{l+s+s1}{\PYZsq{}}\PY{l+s+s1}{hor\PYZus{}mass\PYZus{}center\PYZus{}cat}\PY{l+s+s1}{\PYZsq{}}\PY{p}{,} \PY{l+s+s1}{\PYZsq{}}\PY{l+s+s1}{vert\PYZus{}symmetry\PYZus{}cat}\PY{l+s+s1}{\PYZsq{}}\PY{p}{)}\PY{p}{:}
    \PY{k}{for} \PY{n}{confidence} \PY{o+ow}{in} \PY{p}{[}\PY{l+m+mf}{0.95}\PY{p}{,} \PY{l+m+mf}{0.99}\PY{p}{]}\PY{p}{:}
        \PY{n}{min\PYZus{}N} \PY{o}{=} \PY{p}{(}\PY{n}{chi2}\PY{p}{(}\PY{n}{df}\PY{o}{=}\PY{l+m+mi}{2}\PY{p}{)}\PY{o}{.}\PY{n}{ppf}\PY{p}{(}\PY{n}{confidence}\PY{p}{)} \PY{o}{/} \PY{n}{chi\PYZus{}square}\PY{p}{(}\PY{n}{train}\PY{p}{,} \PY{n}{feature1}\PY{p}{,} \PY{n}{feature2}\PY{p}{)}\PY{p}{[}\PY{l+m+mi}{0}\PY{p}{]}\PY{p}{)}
        \PY{n+nb}{print}\PY{p}{(}\PY{l+s+s1}{\PYZsq{}}\PY{l+s+si}{\PYZob{}\PYZcb{}}\PY{l+s+s1}{ and }\PY{l+s+si}{\PYZob{}\PYZcb{}}\PY{l+s+s1}{ are associated with confidence }\PY{l+s+si}{\PYZob{}:.2f\PYZcb{}}\PY{l+s+se}{\PYZbs{}n}\PY{l+s+se}{\PYZbs{}t}\PY{l+s+s1}{when N \PYZgt{}= }\PY{l+s+si}{\PYZob{}:.1f\PYZcb{}}\PY{l+s+s1}{\PYZsq{}}\PY{o}{.}\PY{n}{format}\PY{p}{(}
            \PY{n}{feature1}\PY{p}{,} \PY{n}{feature2}\PY{p}{,} \PY{n}{confidence}\PY{p}{,} \PY{n}{min\PYZus{}N}
        \PY{p}{)}\PY{p}{)}
\end{Verbatim}
\end{tcolorbox}

    \begin{Verbatim}[commandchars=\\\{\}]
count\_cat and hor\_mass\_center\_cat are associated with confidence 0.95
        when N >= 991.7
count\_cat and hor\_mass\_center\_cat are associated with confidence 0.99
        when N >= 1524.5
count\_cat and vert\_symmetry\_cat are associated with confidence 0.95
        when N >= 98.8
count\_cat and vert\_symmetry\_cat are associated with confidence 0.99
        when N >= 151.8
    \end{Verbatim}

    In our case we have \(1000\) of samples and we can conlude, that we have
enough data to say that the vert\_symmetry depends on count of pixels
with \(99\%\) confidence, and enough data to say that ho\_symmetry
depends on count of pixels with \(95\%\) confidence (but not \(99\%\)
confidence).


\section*{PCA/SVD}\label{pcasvd}

\subsection*{Features}

In this part we have get such features, as \textbf{mean}, \textbf{count}, \textbf{vert\_symmetry}.
We choose this features, because  \textbf{mean}, \textbf{count} high correlated and interesting, how good pca and svd decorelate this features.
        
    \begin{tcolorbox}[breakable, size=fbox, boxrule=1pt, pad at break*=1mm,colback=cellbackground, colframe=cellborder]
\prompt{In}{incolor}{69}{\boxspacing}
\begin{Verbatim}[commandchars=\\\{\}]
\PY{n}{features} \PY{o}{=} \PY{p}{[}\PY{l+s+s1}{\PYZsq{}}\PY{l+s+s1}{mean}\PY{l+s+s1}{\PYZsq{}}\PY{p}{,} \PY{l+s+s1}{\PYZsq{}}\PY{l+s+s1}{count}\PY{l+s+s1}{\PYZsq{}}\PY{p}{,} \PY{l+s+s1}{\PYZsq{}}\PY{l+s+s1}{vert\PYZus{}symmetry}\PY{l+s+s1}{\PYZsq{}}\PY{p}{]}
\end{Verbatim}
\end{tcolorbox}

    \begin{tcolorbox}[breakable, size=fbox, boxrule=1pt, pad at break*=1mm,colback=cellbackground, colframe=cellborder]
\prompt{In}{incolor}{70}{\boxspacing}
\begin{Verbatim}[commandchars=\\\{\}]
\PY{n}{X\PYZus{}train} \PY{o}{=} \PY{n}{np}\PY{o}{.}\PY{n}{array}\PY{p}{(}\PY{p}{[}\PY{n}{train}\PY{p}{[}\PY{n}{feature}\PY{p}{]}\PY{o}{.}\PY{n}{values} \PY{k}{for} \PY{n}{feature} \PY{o+ow}{in} \PY{n}{features}\PY{p}{]}\PY{p}{)}\PY{o}{.}\PY{n}{T}
\PY{n}{X\PYZus{}test} \PY{o}{=} \PY{n}{np}\PY{o}{.}\PY{n}{array}\PY{p}{(}\PY{p}{[}\PY{n}{test}\PY{p}{[}\PY{n}{feature}\PY{p}{]}\PY{o}{.}\PY{n}{values} \PY{k}{for} \PY{n}{feature} \PY{o+ow}{in} \PY{n}{features}\PY{p}{]}\PY{p}{)}\PY{o}{.}\PY{n}{T}
\end{Verbatim}
\end{tcolorbox}

\subsection*{Standardize and SVD}

    \begin{tcolorbox}[breakable, size=fbox, boxrule=1pt, pad at break*=1mm,colback=cellbackground, colframe=cellborder]
\prompt{In}{incolor}{71}{\boxspacing}
\begin{Verbatim}[commandchars=\\\{\}]
\PY{n}{means} \PY{o}{=} \PY{n}{np}\PY{o}{.}\PY{n}{mean}\PY{p}{(}\PY{n}{X\PYZus{}train}\PY{p}{,} \PY{n}{axis}\PY{o}{=}\PY{l+m+mi}{0}\PY{p}{,} \PY{n}{keepdims}\PY{o}{=}\PY{k+kc}{True}\PY{p}{)}
\PY{n}{stds} \PY{o}{=} \PY{n}{np}\PY{o}{.}\PY{n}{std}\PY{p}{(}\PY{n}{X\PYZus{}train}\PY{p}{,} \PY{n}{axis}\PY{o}{=}\PY{l+m+mi}{0}\PY{p}{,} \PY{n}{keepdims}\PY{o}{=}\PY{k+kc}{True}\PY{p}{)}
\PY{n}{X\PYZus{}train\PYZus{}norm} \PY{o}{=} \PY{p}{(}\PY{n}{X\PYZus{}train} \PY{o}{\PYZhy{}} \PY{n}{means}\PY{p}{)} \PY{o}{/} \PY{n}{stds}

\PY{n+nb}{print}\PY{p}{(}\PY{l+s+s1}{\PYZsq{}}\PY{l+s+s1}{data scatter:}\PY{l+s+s1}{\PYZsq{}}\PY{p}{,} \PY{n}{np}\PY{o}{.}\PY{n}{sum}\PY{p}{(}\PY{n}{X\PYZus{}train} \PY{o}{*}\PY{o}{*} \PY{l+m+mi}{2}\PY{p}{)}\PY{p}{)}
\PY{n+nb}{print}\PY{p}{(}\PY{l+s+s1}{\PYZsq{}}\PY{l+s+s1}{data scatter after centering: }\PY{l+s+s1}{\PYZsq{}}\PY{p}{,} \PY{n}{np}\PY{o}{.}\PY{n}{sum}\PY{p}{(}\PY{p}{(}\PY{n}{X\PYZus{}train} \PY{o}{\PYZhy{}} \PY{n}{means}\PY{p}{)} \PY{o}{*}\PY{o}{*} \PY{l+m+mi}{2}\PY{p}{)}\PY{p}{)}
\PY{n+nb}{print}\PY{p}{(}\PY{l+s+s1}{\PYZsq{}}\PY{l+s+s1}{data scatter after standardize: }\PY{l+s+s1}{\PYZsq{}}\PY{p}{,} \PY{n}{np}\PY{o}{.}\PY{n}{sum}\PY{p}{(}\PY{n}{X\PYZus{}train\PYZus{}norm} \PY{o}{*}\PY{o}{*} \PY{l+m+mi}{2}\PY{p}{)}\PY{p}{)}
\end{Verbatim}
\end{tcolorbox}

    \begin{Verbatim}[commandchars=\\\{\}]
data scatter: 25337027.391464494
data scatter after centering:  1998165.1463920956
data scatter after standardize:  2999.9999999999995
    \end{Verbatim}

    \begin{tcolorbox}[breakable, size=fbox, boxrule=1pt, pad at break*=1mm,colback=cellbackground, colframe=cellborder]
\prompt{In}{incolor}{74}{\boxspacing}
\begin{Verbatim}[commandchars=\\\{\}]
\PY{n}{pca} \PY{o}{=} \PY{n}{PCA}\PY{p}{(}\PY{p}{)}
\PY{n}{transformed\PYZus{}array} \PY{o}{=} \PY{n}{pca}\PY{o}{.}\PY{n}{fit\PYZus{}transform}\PY{p}{(}\PY{n}{X\PYZus{}train\PYZus{}norm}\PY{p}{)}
\PY{n}{transformed} \PY{o}{=} \PY{n}{pd}\PY{o}{.}\PY{n}{DataFrame}\PY{p}{(}\PY{n}{transformed\PYZus{}array}\PY{p}{,} \PY{n}{columns}\PY{o}{=}\PY{p}{[}\PY{l+s+s1}{\PYZsq{}}\PY{l+s+s1}{PC}\PY{l+s+s1}{\PYZsq{}} \PY{o}{+} \PY{n+nb}{str}\PY{p}{(}\PY{n}{i}\PY{p}{)} \PY{k}{for} \PY{n}{i} \PY{o+ow}{in} \PY{n+nb}{range}\PY{p}{(}\PY{l+m+mi}{1}\PY{p}{,} \PY{n+nb}{len}\PY{p}{(}\PY{n}{features}\PY{p}{)} \PY{o}{+} \PY{l+m+mi}{1}\PY{p}{)}\PY{p}{]}\PY{p}{)}
\end{Verbatim}
\end{tcolorbox}

    \begin{tcolorbox}[breakable, size=fbox, boxrule=1pt, pad at break*=1mm,colback=cellbackground, colframe=cellborder]
\prompt{In}{incolor}{75}{\boxspacing}
\begin{Verbatim}[commandchars=\\\{\}]
\PY{n}{transformed}\PY{o}{.}\PY{n}{head}\PY{p}{(}\PY{p}{)}
\end{Verbatim}
\end{tcolorbox}

            \begin{tcolorbox}[breakable, size=fbox, boxrule=.5pt, pad at break*=1mm, opacityfill=0]
\prompt{Out}{outcolor}{75}{\boxspacing}
\begin{Verbatim}[commandchars=\\\{\}]
        PC1       PC2       PC3
0 -1.317348  1.511233  0.042066
1  0.479915  0.407563  0.375165
2  0.757433  0.808916  0.052019
3  0.324062  0.374046  0.216130
4  1.857682  0.182502  0.009792
\end{Verbatim}
\end{tcolorbox}
        
    \begin{tcolorbox}[breakable, size=fbox, boxrule=1pt, pad at break*=1mm,colback=cellbackground, colframe=cellborder]
\prompt{In}{incolor}{76}{\boxspacing}
\begin{Verbatim}[commandchars=\\\{\}]
\PY{n+nb}{print}\PY{p}{(}\PY{l+s+s1}{\PYZsq{}}\PY{l+s+s1}{pca components: }\PY{l+s+s1}{\PYZsq{}}\PY{p}{,} \PY{n}{pca}\PY{o}{.}\PY{n}{components\PYZus{}}\PY{p}{)}
\end{Verbatim}
\end{tcolorbox}

    \begin{Verbatim}[commandchars=\\\{\}]
pca components:  [[ 0.68032123  0.68058897 -0.27195897]
 [ 0.19338644  0.19122157  0.96230764]
 [ 0.70694038 -0.7072715  -0.00152458]]
    \end{Verbatim}

    \begin{tcolorbox}[breakable, size=fbox, boxrule=1pt, pad at break*=1mm,colback=cellbackground, colframe=cellborder]
\prompt{In}{incolor}{77}{\boxspacing}
\begin{Verbatim}[commandchars=\\\{\}]
\PY{n}{scatter} \PY{o}{=} \PY{n}{np}\PY{o}{.}\PY{n}{sum}\PY{p}{(}\PY{n}{X\PYZus{}train\PYZus{}norm} \PY{o}{*}\PY{o}{*} \PY{l+m+mi}{2}\PY{p}{)}
\PY{k}{for} \PY{n}{col\PYZus{}name} \PY{o+ow}{in} \PY{n}{transformed}\PY{p}{:}
    \PY{n}{col\PYZus{}scatter} \PY{o}{=} \PY{n}{np}\PY{o}{.}\PY{n}{sum}\PY{p}{(}\PY{n}{transformed}\PY{p}{[}\PY{n}{col\PYZus{}name}\PY{p}{]} \PY{o}{*}\PY{o}{*} \PY{l+m+mi}{2}\PY{p}{)}
    \PY{n}{scatter\PYZus{}percent} \PY{o}{=} \PY{l+m+mi}{100} \PY{o}{*} \PY{n}{col\PYZus{}scatter} \PY{o}{/} \PY{n}{scatter}
    \PY{n+nb}{print}\PY{p}{(}\PY{n}{col\PYZus{}name}\PY{p}{,} \PY{l+s+s1}{\PYZsq{}}\PY{l+s+s1}{contributes }\PY{l+s+si}{\PYZob{}0:.3f\PYZcb{}}\PY{l+s+s1}{, or }\PY{l+s+si}{\PYZob{}1:.2f\PYZcb{}}\PY{l+s+s1}{\PYZpc{}}\PY{l+s+s1}{, to the data scatter}\PY{l+s+s1}{\PYZsq{}}\PY{o}{.}\PY{n}{format}\PY{p}{(}\PY{n}{col\PYZus{}scatter}\PY{p}{,} \PY{n}{scatter\PYZus{}percent}\PY{p}{)}\PY{p}{)}
\end{Verbatim}
\end{tcolorbox}

    \begin{Verbatim}[commandchars=\\\{\}]
PC1 contributes 2046.691, or 68.22\%, to the data scatter
PC2 contributes 916.404, or 30.55\%, to the data scatter
PC3 contributes 36.905, or 1.23\%, to the data scatter
    \end{Verbatim}

\subsection*{Hidden ranking factor}

To obtain a hidden factor expressed in the 0-100 rank scale, we first rescale all the features to this range and then decompose the result using SVD and output the first component.

    \begin{tcolorbox}[breakable, size=fbox, boxrule=1pt, pad at break*=1mm,colback=cellbackground, colframe=cellborder]
\prompt{In}{incolor}{78}{\boxspacing}
\begin{Verbatim}[commandchars=\\\{\}]
\PY{k}{def} \PY{n+nf}{rescale}\PY{p}{(}\PY{n}{df}\PY{p}{)}\PY{p}{:}
    \PY{k}{return} \PY{p}{(}\PY{n}{df} \PY{o}{\PYZhy{}} \PY{n}{df}\PY{o}{.}\PY{n}{min}\PY{p}{(}\PY{p}{)}\PY{p}{)} \PY{o}{/} \PY{p}{(}\PY{n}{df}\PY{o}{.}\PY{n}{max}\PY{p}{(}\PY{p}{)} \PY{o}{\PYZhy{}} \PY{n}{df}\PY{o}{.}\PY{n}{min}\PY{p}{(}\PY{p}{)}\PY{p}{)} \PY{o}{*} \PY{l+m+mi}{100}

\PY{n}{rescaled\PYZus{}X} \PY{o}{=} \PY{n}{rescale}\PY{p}{(}\PY{n}{X\PYZus{}train}\PY{p}{)}
\PY{n}{U}\PY{p}{,} \PY{n}{s}\PY{p}{,} \PY{n}{V} \PY{o}{=} \PY{n}{np}\PY{o}{.}\PY{n}{linalg}\PY{o}{.}\PY{n}{svd}\PY{p}{(}\PY{n}{rescaled\PYZus{}X}\PY{p}{)}
\PY{n}{contribution} \PY{o}{=} \PY{l+m+mi}{100} \PY{o}{*} \PY{n}{s}\PY{p}{[}\PY{l+m+mi}{0}\PY{p}{]} \PY{o}{*}\PY{o}{*} \PY{l+m+mi}{2} \PY{o}{/} \PY{n}{np}\PY{o}{.}\PY{n}{sum}\PY{p}{(}\PY{n}{rescaled\PYZus{}X} \PY{o}{*}\PY{o}{*} \PY{l+m+mi}{2}\PY{p}{)}
\end{Verbatim}
\end{tcolorbox}

    \begin{tcolorbox}[breakable, size=fbox, boxrule=1pt, pad at break*=1mm,colback=cellbackground, colframe=cellborder]
\prompt{In}{incolor}{79}{\boxspacing}
\begin{Verbatim}[commandchars=\\\{\}]
\PY{n+nb}{print}\PY{p}{(}\PY{l+s+s1}{\PYZsq{}}\PY{l+s+s1}{First component:}\PY{l+s+s1}{\PYZsq{}}\PY{p}{,} \PY{n}{V}\PY{p}{[}\PY{l+m+mi}{0}\PY{p}{]}\PY{p}{)}
\PY{n+nb}{print}\PY{p}{(}\PY{l+s+s1}{\PYZsq{}}\PY{l+s+s1}{Its contribution to the data scatter: }\PY{l+s+si}{\PYZob{}0:.3f\PYZcb{}}\PY{l+s+s1}{\PYZpc{}}\PY{l+s+s1}{\PYZsq{}}\PY{o}{.}\PY{n}{format}\PY{p}{(}\PY{n}{contribution}\PY{p}{)}\PY{p}{)}
\end{Verbatim}
\end{tcolorbox}

    \begin{Verbatim}[commandchars=\\\{\}]
First component: [-0.33506822 -0.93161987 -0.14076113]
Its contribution to the data scatter: 99.638\%
    \end{Verbatim}
    
   That the contribution of the first component is much higher than in the previous case should not be surprising: the data is not centered, so its mean is the source of most of the data scatter.

\subsection*{Visualization}

At first visualize all labels.

    \begin{tcolorbox}[breakable, size=fbox, boxrule=1pt, pad at break*=1mm,colback=cellbackground, colframe=cellborder]
\prompt{In}{incolor}{99}{\boxspacing}
\begin{Verbatim}[commandchars=\\\{\}]
\PY{k}{def} \PY{n+nf}{pairplot}\PY{p}{(}\PY{n}{df}\PY{p}{,} \PY{n}{title}\PY{p}{)}\PY{p}{:}
    \PY{n}{plot} \PY{o}{=} \PY{n}{sns}\PY{o}{.}\PY{n}{pairplot}\PY{p}{(}\PY{n}{df}\PY{p}{,} \PY{n+nb}{vars}\PY{o}{=}\PY{n}{df}\PY{o}{.}\PY{n}{columns}\PY{p}{[}\PY{p}{:}\PY{o}{\PYZhy{}}\PY{l+m+mi}{1}\PY{p}{]}\PY{p}{,} \PY{n}{hue}\PY{o}{=}\PY{l+s+s1}{\PYZsq{}}\PY{l+s+s1}{label}\PY{l+s+s1}{\PYZsq{}}\PY{p}{)}
    \PY{n}{plot}\PY{o}{.}\PY{n}{fig}\PY{o}{.}\PY{n}{suptitle}\PY{p}{(}\PY{n}{title}\PY{p}{)}

\PY{n}{df\PYZus{}standart} \PY{o}{=} \PY{n}{pd}\PY{o}{.}\PY{n}{DataFrame}\PY{p}{(}\PY{n}{X\PYZus{}train\PYZus{}norm}\PY{p}{,} \PY{n}{columns}\PY{o}{=}\PY{n}{features}\PY{p}{)} 
\PY{n}{df\PYZus{}rescale} \PY{o}{=} \PY{n}{pd}\PY{o}{.}\PY{n}{DataFrame}\PY{p}{(}\PY{n}{rescaled\PYZus{}X}\PY{p}{,} \PY{n}{columns}\PY{o}{=}\PY{n}{features}\PY{p}{)}
\PY{n}{df\PYZus{}transform} \PY{o}{=} \PY{n}{pd}\PY{o}{.}\PY{n}{DataFrame}\PY{p}{(}\PY{n}{transformed\PYZus{}array}\PY{p}{,} \PY{n}{columns}\PY{o}{=}\PY{p}{[}\PY{l+s+s1}{\PYZsq{}}\PY{l+s+s1}{PC}\PY{l+s+s1}{\PYZsq{}} \PY{o}{+} \PY{n+nb}{str}\PY{p}{(}\PY{n}{i}\PY{p}{)} \PY{k}{for} \PY{n}{i} \PY{o+ow}{in} \PY{n+nb}{range}\PY{p}{(}\PY{l+m+mi}{1}\PY{p}{,} \PY{n+nb}{len}\PY{p}{(}\PY{n}{features}\PY{p}{)} \PY{o}{+} \PY{l+m+mi}{1}\PY{p}{)}\PY{p}{]}\PY{p}{)}

\PY{k}{for} \PY{n}{df} \PY{o+ow}{in} \PY{p}{(}\PY{n}{df\PYZus{}standart}\PY{p}{,} \PY{n}{df\PYZus{}rescale}\PY{p}{,} \PY{n}{df\PYZus{}transform}\PY{p}{)}\PY{p}{:}
    \PY{n}{df}\PY{p}{[}\PY{l+s+s1}{\PYZsq{}}\PY{l+s+s1}{label}\PY{l+s+s1}{\PYZsq{}}\PY{p}{]} \PY{o}{=} \PY{n}{train}\PY{p}{[}\PY{l+s+s1}{\PYZsq{}}\PY{l+s+s1}{label}\PY{l+s+s1}{\PYZsq{}}\PY{p}{]}

\PY{n}{pairplot}\PY{p}{(}\PY{n}{df\PYZus{}standart}\PY{p}{,} \PY{l+s+s1}{\PYZsq{}}\PY{l+s+s1}{Standardized (mean=0, std=1)}\PY{l+s+s1}{\PYZsq{}}\PY{p}{)}
\PY{n}{pairplot}\PY{p}{(}\PY{n}{df\PYZus{}rescale}\PY{p}{,} \PY{l+s+s1}{\PYZsq{}}\PY{l+s+s1}{Normalized (range=[0, 100])}\PY{l+s+s1}{\PYZsq{}}\PY{p}{)}
\PY{n}{pairplot}\PY{p}{(}\PY{n}{df\PYZus{}transform}\PY{p}{,} \PY{l+s+s1}{\PYZsq{}}\PY{l+s+s1}{Primary components}\PY{l+s+s1}{\PYZsq{}}\PY{p}{)}
\end{Verbatim}
\end{tcolorbox}

    \begin{center}
    \adjustimage{max size={0.9\linewidth}{0.9\paperheight}}{PCA-SVD/output_17_0.png}
    \end{center}
    { \hspace*{\fill} \\}
    
    \begin{center}
    \adjustimage{max size={0.9\linewidth}{0.9\paperheight}}{PCA-SVD/output_17_1.png}
    \end{center}
    { \hspace*{\fill} \\}
    
    \begin{center}
    \adjustimage{max size={0.9\linewidth}{0.9\paperheight}}{PCA-SVD/output_17_2.png}
    \end{center}
    { \hspace*{\fill} \\}
    
 Too difficult understand something in case with many labels. Visualize only first 4 digits.
    
    \begin{tcolorbox}[breakable, size=fbox, boxrule=1pt, pad at break*=1mm,colback=cellbackground, colframe=cellborder]
\prompt{In}{incolor}{100}{\boxspacing}
\begin{Verbatim}[commandchars=\\\{\}]
\PY{n}{indexes} \PY{o}{=} \PY{n}{train}\PY{p}{[}\PY{l+s+s1}{\PYZsq{}}\PY{l+s+s1}{label}\PY{l+s+s1}{\PYZsq{}}\PY{p}{]}\PY{o}{.}\PY{n}{values} \PY{o}{\PYZlt{}}\PY{o}{=} \PY{l+m+mi}{3}
\PY{n}{df\PYZus{}standart}\PY{p}{,} \PY{n}{df\PYZus{}rescale}\PY{p}{,} \PY{n}{df\PYZus{}transform} \PY{o}{=} \PY{p}{[}\PY{n}{df}\PY{p}{[}\PY{n}{indexes}\PY{p}{]} \PY{k}{for} \PY{n}{df} \PY{o+ow}{in} \PY{p}{(}\PY{n}{df\PYZus{}standart}\PY{p}{,} \PY{n}{df\PYZus{}rescale}\PY{p}{,} \PY{n}{df\PYZus{}transform}\PY{p}{)}\PY{p}{]}

\PY{n}{pairplot}\PY{p}{(}\PY{n}{df\PYZus{}standart}\PY{p}{,} \PY{l+s+s1}{\PYZsq{}}\PY{l+s+s1}{Standardized (mean=0, std=1)}\PY{l+s+s1}{\PYZsq{}}\PY{p}{)}
\PY{n}{pairplot}\PY{p}{(}\PY{n}{df\PYZus{}rescale}\PY{p}{,} \PY{l+s+s1}{\PYZsq{}}\PY{l+s+s1}{Normalized (range=[0, 100])}\PY{l+s+s1}{\PYZsq{}}\PY{p}{)}
\PY{n}{pairplot}\PY{p}{(}\PY{n}{df\PYZus{}transform}\PY{p}{,} \PY{l+s+s1}{\PYZsq{}}\PY{l+s+s1}{Primary components}\PY{l+s+s1}{\PYZsq{}}\PY{p}{)}
\end{Verbatim}
\end{tcolorbox}

    \begin{center}
    \adjustimage{max size={0.9\linewidth}{0.9\paperheight}}{PCA-SVD/output_18_0.png}
    \end{center}
    { \hspace*{\fill} \\}
    
    \begin{center}
    \adjustimage{max size={0.9\linewidth}{0.9\paperheight}}{PCA-SVD/output_18_1.png}
    \end{center}
    { \hspace*{\fill} \\}
    
    \begin{center}
    \adjustimage{max size={0.9\linewidth}{0.9\paperheight}}{PCA-SVD/output_18_2.png}
    \end{center}
    { \hspace*{\fill} \\}
    
    Obviously, since we only use line plots and pairwise scatter plots, the difference between standardization and normalization amounts to relabeling of the axes (which might be helpful by itself because it makes interpreting the coordinates in the graph conceptually easier).
    
    However, if we were, for example, using a 3D scatter plot with fixed scales of the axes, normalization into the [0, 1] range would help a lot as it guarantees that the features have the same scale (if there are no outliers).
    
    PCA doesn’t seem to help a lot with distinguishing the points with digits labeling. All features can divede ones from others class, but with other digits all features have problems. But PC2 can divide all digits on three gropus: \{2, 4, 6\}, \{1\}, \{0, 3, 5, 7, 8, 9\}


\section*{Correlation coefficient}
\subsection*{Features}
We have get two features mean and count, because this features high correlated. Mean it's sum of values of all pixels on image. Count it's count of nonzero pixels on image.
    \begin{tcolorbox}[breakable, size=fbox, boxrule=1pt, pad at break*=1mm,colback=cellbackground, colframe=cellborder]
\prompt{In}{incolor}{5}{\boxspacing}
\begin{Verbatim}[commandchars=\\\{\}]
\PY{n}{x\PYZus{}feature} \PY{o}{=} \PY{l+s+s1}{\PYZsq{}}\PY{l+s+s1}{mean}\PY{l+s+s1}{\PYZsq{}}
\PY{n}{y\PYZus{}feature} \PY{o}{=} \PY{l+s+s1}{\PYZsq{}}\PY{l+s+s1}{count}\PY{l+s+s1}{\PYZsq{}}
\end{Verbatim}
\end{tcolorbox}

    \begin{tcolorbox}[breakable, size=fbox, boxrule=1pt, pad at break*=1mm,colback=cellbackground, colframe=cellborder]
\prompt{In}{incolor}{6}{\boxspacing}
\begin{Verbatim}[commandchars=\\\{\}]
\PY{n}{X\PYZus{}train} \PY{o}{=} \PY{n}{train}\PY{p}{[}\PY{n}{x\PYZus{}feature}\PY{p}{]}\PY{o}{.}\PY{n}{values}
\PY{n}{X\PYZus{}test} \PY{o}{=} \PY{n}{test}\PY{p}{[}\PY{n}{x\PYZus{}feature}\PY{p}{]}\PY{o}{.}\PY{n}{values}
\PY{n}{Y\PYZus{}train} \PY{o}{=} \PY{n}{train}\PY{p}{[}\PY{n}{y\PYZus{}feature}\PY{p}{]}\PY{o}{.}\PY{n}{values}
\PY{n}{Y\PYZus{}test} \PY{o}{=} \PY{n}{test}\PY{p}{[}\PY{n}{y\PYZus{}feature}\PY{p}{]}\PY{o}{.}\PY{n}{values}
\end{Verbatim}
\end{tcolorbox}

\subsection*{Visualize}
    \begin{tcolorbox}[breakable, size=fbox, boxrule=1pt, pad at break*=1mm,colback=cellbackground, colframe=cellborder]
\prompt{In}{incolor}{7}{\boxspacing}
\begin{Verbatim}[commandchars=\\\{\}]
\PY{n}{plt}\PY{o}{.}\PY{n}{plot}\PY{p}{(}\PY{n}{X\PYZus{}train}\PY{p}{,} \PY{n}{Y\PYZus{}train}\PY{p}{,} \PY{l+s+s1}{\PYZsq{}}\PY{l+s+s1}{o}\PY{l+s+s1}{\PYZsq{}}\PY{p}{)}
\PY{n}{plt}\PY{o}{.}\PY{n}{show}\PY{p}{(}\PY{p}{)}
\end{Verbatim}
\end{tcolorbox}

    \begin{center}
    \adjustimage{max size={0.9\linewidth}{0.9\paperheight}}{correlation-coefficient/output_6_0.png}
    \end{center}
    { \hspace*{\fill} \\}

\subsection*{Linear Regression}

    \begin{tcolorbox}[breakable, size=fbox, boxrule=1pt, pad at break*=1mm,colback=cellbackground, colframe=cellborder]
\prompt{In}{incolor}{8}{\boxspacing}
\begin{Verbatim}[commandchars=\\\{\}]
\PY{n}{model} \PY{o}{=} \PY{n}{LinearRegression}\PY{p}{(}\PY{p}{)}
\PY{n}{model}\PY{o}{.}\PY{n}{fit}\PY{p}{(}\PY{n}{X\PYZus{}train}\PY{o}{.}\PY{n}{reshape}\PY{p}{(}\PY{o}{\PYZhy{}}\PY{l+m+mi}{1}\PY{p}{,} \PY{l+m+mi}{1}\PY{p}{)}\PY{p}{,} \PY{n}{Y\PYZus{}train}\PY{o}{.}\PY{n}{reshape}\PY{p}{(}\PY{o}{\PYZhy{}}\PY{l+m+mi}{1}\PY{p}{,} \PY{l+m+mi}{1}\PY{p}{)}\PY{p}{)}
\end{Verbatim}
\end{tcolorbox}

            \begin{tcolorbox}[breakable, size=fbox, boxrule=.5pt, pad at break*=1mm, opacityfill=0]
\prompt{Out}{outcolor}{8}{\boxspacing}
\begin{Verbatim}[commandchars=\\\{\}]
LinearRegression(copy\_X=True, fit\_intercept=True, n\_jobs=None, normalize=False)
\end{Verbatim}
\end{tcolorbox}
        
    \begin{tcolorbox}[breakable, size=fbox, boxrule=1pt, pad at break*=1mm,colback=cellbackground, colframe=cellborder]
\prompt{In}{incolor}{9}{\boxspacing}
\begin{Verbatim}[commandchars=\\\{\}]
\PY{n}{model}\PY{o}{.}\PY{n}{coef\PYZus{}}
\end{Verbatim}
\end{tcolorbox}

            \begin{tcolorbox}[breakable, size=fbox, boxrule=.5pt, pad at break*=1mm, opacityfill=0]
\prompt{Out}{outcolor}{9}{\boxspacing}
\begin{Verbatim}[commandchars=\\\{\}]
array([[3.5836792]])
\end{Verbatim}
\end{tcolorbox}

We have coef of linear regression equal to 3.58, that mean, that count of pixels in 3.58 times greater than mean. We know, that mean -- it's sum of all pixels divided by $28 \times 28 = 784$. And we know, that large part of pixels equal to $255$. It means, that count approximately equal to sum of all pixels divided by $255$. In such way we know, that koef $.3.58$ hight correlated with out knowledge.

\subsection*{Correlation and determinacy}
        
    \begin{tcolorbox}[breakable, size=fbox, boxrule=1pt, pad at break*=1mm,colback=cellbackground, colframe=cellborder]
\prompt{In}{incolor}{10}{\boxspacing}
\begin{Verbatim}[commandchars=\\\{\}]
\PY{n}{x\PYZus{}mean} \PY{o}{=} \PY{n}{np}\PY{o}{.}\PY{n}{mean}\PY{p}{(}\PY{n}{X\PYZus{}train}\PY{p}{)}
\PY{n}{y\PYZus{}mean} \PY{o}{=} \PY{n}{np}\PY{o}{.}\PY{n}{mean}\PY{p}{(}\PY{n}{Y\PYZus{}train}\PY{p}{)}
\PY{n}{cov} \PY{o}{=} \PY{n}{np}\PY{o}{.}\PY{n}{sum}\PY{p}{(}\PY{p}{(}\PY{n}{X\PYZus{}train} \PY{o}{\PYZhy{}} \PY{n}{x\PYZus{}mean}\PY{p}{)} \PY{o}{*} \PY{p}{(}\PY{n}{Y\PYZus{}train} \PY{o}{\PYZhy{}} \PY{n}{y\PYZus{}mean}\PY{p}{)}\PY{p}{)}
\PY{n}{x\PYZus{}var} \PY{o}{=} \PY{n}{np}\PY{o}{.}\PY{n}{sum}\PY{p}{(}\PY{p}{(}\PY{n}{X\PYZus{}train} \PY{o}{\PYZhy{}} \PY{n}{x\PYZus{}mean}\PY{p}{)} \PY{o}{*}\PY{o}{*} \PY{l+m+mi}{2}\PY{p}{)}
\PY{n}{y\PYZus{}var} \PY{o}{=} \PY{n}{np}\PY{o}{.}\PY{n}{sum}\PY{p}{(}\PY{p}{(}\PY{n}{Y\PYZus{}train} \PY{o}{\PYZhy{}} \PY{n}{y\PYZus{}mean}\PY{p}{)} \PY{o}{*}\PY{o}{*} \PY{l+m+mi}{2}\PY{p}{)}
\PY{n}{cor} \PY{o}{=} \PY{n}{cov} \PY{o}{/} \PY{n}{np}\PY{o}{.}\PY{n}{sqrt}\PY{p}{(}\PY{n}{x\PYZus{}var} \PY{o}{*} \PY{n}{y\PYZus{}var}\PY{p}{)}
\PY{n+nb}{print}\PY{p}{(}\PY{l+s+s1}{\PYZsq{}}\PY{l+s+s1}{correlation coefficient:}\PY{l+s+s1}{\PYZsq{}}\PY{p}{,} \PY{n}{cor}\PY{p}{)}
\end{Verbatim}
\end{tcolorbox}

    \begin{Verbatim}[commandchars=\\\{\}]
correlation coefficient: 0.9630931048114645
    \end{Verbatim}

It means, that features high correlated, such as shown on picture above.

    \begin{tcolorbox}[breakable, size=fbox, boxrule=1pt, pad at break*=1mm,colback=cellbackground, colframe=cellborder]
\prompt{In}{incolor}{11}{\boxspacing}
\begin{Verbatim}[commandchars=\\\{\}]
\PY{n}{y\PYZus{}mean} \PY{o}{=} \PY{n}{np}\PY{o}{.}\PY{n}{mean}\PY{p}{(}\PY{n}{Y\PYZus{}test}\PY{p}{)}
\PY{n}{y\PYZus{}var} \PY{o}{=} \PY{n}{np}\PY{o}{.}\PY{n}{sum}\PY{p}{(}\PY{p}{(}\PY{n}{Y\PYZus{}test} \PY{o}{\PYZhy{}} \PY{n}{y\PYZus{}mean}\PY{p}{)} \PY{o}{*}\PY{o}{*} \PY{l+m+mi}{2}\PY{p}{)}
\PY{n}{pred\PYZus{}y} \PY{o}{=} \PY{n}{model}\PY{o}{.}\PY{n}{predict}\PY{p}{(}\PY{n}{X\PYZus{}test}\PY{o}{.}\PY{n}{reshape}\PY{p}{(}\PY{o}{\PYZhy{}}\PY{l+m+mi}{1}\PY{p}{,} \PY{l+m+mi}{1}\PY{p}{)}\PY{p}{)}\PY{o}{.}\PY{n}{reshape}\PY{p}{(}\PY{o}{\PYZhy{}}\PY{l+m+mi}{1}\PY{p}{)}
\PY{n}{pred\PYZus{}var} \PY{o}{=} \PY{n}{np}\PY{o}{.}\PY{n}{sum}\PY{p}{(}\PY{p}{(}\PY{n}{pred\PYZus{}y} \PY{o}{\PYZhy{}} \PY{n}{Y\PYZus{}test}\PY{p}{)} \PY{o}{*}\PY{o}{*} \PY{l+m+mi}{2}\PY{p}{)}
\PY{n}{det} \PY{o}{=} \PY{l+m+mi}{1} \PY{o}{\PYZhy{}} \PY{n}{pred\PYZus{}var} \PY{o}{/} \PY{n}{y\PYZus{}var}
\PY{n+nb}{print}\PY{p}{(}\PY{l+s+s1}{\PYZsq{}}\PY{l+s+s1}{determinacy coefficient:}\PY{l+s+s1}{\PYZsq{}}\PY{p}{,} \PY{n}{det}\PY{p}{)}
\end{Verbatim}
\end{tcolorbox}

    \begin{Verbatim}[commandchars=\\\{\}]
determinacy coefficient: 0.9234620267287004
    \end{Verbatim}

It means, that large part of variance of count variable explains by mean variable.

\subsection*{Prediction sample}

Predict three samples from test.

    \begin{tcolorbox}[breakable, size=fbox, boxrule=1pt, pad at break*=1mm,colback=cellbackground, colframe=cellborder]
\prompt{In}{incolor}{36}{\boxspacing}
\begin{Verbatim}[commandchars=\\\{\}]
\PY{n+nb}{print}\PY{p}{(}\PY{l+s+s1}{\PYZsq{}}\PY{l+s+s1}{Predictions for 3 points:}\PY{l+s+s1}{\PYZsq{}}\PY{p}{)}
\PY{n+nb}{print}\PY{p}{(}\PY{l+s+s1}{\PYZsq{}}\PY{l+s+se}{\PYZbs{}t}\PY{l+s+se}{\PYZbs{}t}\PY{l+s+s1}{y\PYZus{}true}\PY{l+s+se}{\PYZbs{}t}\PY{l+s+se}{\PYZbs{}t}\PY{l+s+s1}{y\PYZus{}pred}\PY{l+s+s1}{\PYZsq{}}\PY{p}{)}
\PY{k}{for} \PY{n}{i} \PY{o+ow}{in} \PY{n+nb}{range}\PY{p}{(}\PY{l+m+mi}{3}\PY{p}{)}\PY{p}{:}
    \PY{n+nb}{print}\PY{p}{(}\PY{l+s+s1}{\PYZsq{}}\PY{l+s+se}{\PYZbs{}t}\PY{l+s+se}{\PYZbs{}t}\PY{l+s+s1}{\PYZsq{}}\PY{p}{,} \PY{n}{Y\PYZus{}test}\PY{p}{[}\PY{n}{i}\PY{p}{]}\PY{p}{,} \PY{l+s+s1}{\PYZsq{}}\PY{l+s+se}{\PYZbs{}t}\PY{l+s+se}{\PYZbs{}t}\PY{l+s+s1}{ }\PY{l+s+si}{\PYZob{}0:.4\PYZcb{}}\PY{l+s+s1}{\PYZsq{}}\PY{o}{.}\PY{n}{format}\PY{p}{(}\PY{n}{pred\PYZus{}y}\PY{p}{[}\PY{n}{i}\PY{p}{]}\PY{p}{)}\PY{p}{)}
\end{Verbatim}
\end{tcolorbox}

    \begin{Verbatim}[commandchars=\\\{\}]
Predictions for 3 points:
                y\_true          y\_pred
                 144             144.3
                 127             118.8
                 188             176.4
    \end{Verbatim}

Result too close to true value.

\subsection*{MRAE}

    \begin{tcolorbox}[breakable, size=fbox, boxrule=1pt, pad at break*=1mm,colback=cellbackground, colframe=cellborder]
\prompt{In}{incolor}{38}{\boxspacing}
\begin{Verbatim}[commandchars=\\\{\}]
\PY{n}{mrae} \PY{o}{=} \PY{n}{np}\PY{o}{.}\PY{n}{mean}\PY{p}{(}\PY{n}{np}\PY{o}{.}\PY{n}{abs}\PY{p}{(}\PY{n}{pred\PYZus{}y} \PY{o}{\PYZhy{}} \PY{n}{Y\PYZus{}test}\PY{p}{)} \PY{o}{/} \PY{n}{Y\PYZus{}test}\PY{p}{)}
\PY{n+nb}{print}\PY{p}{(}\PY{l+s+s1}{\PYZsq{}}\PY{l+s+s1}{Mean relative absolute error:}\PY{l+s+s1}{\PYZsq{}}\PY{p}{,} \PY{n}{mrae}\PY{p}{)}
\end{Verbatim}
\end{tcolorbox}

    \begin{Verbatim}[commandchars=\\\{\}]
Mean relative absolute error: 0.06477857524441351
    \end{Verbatim}

We have small MRAE, that can be explain by determinacy coefficient close to one. That means, that model make relatively good predict on whole span of values for given features.


\end{document}
