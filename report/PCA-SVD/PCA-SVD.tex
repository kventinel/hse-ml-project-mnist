\section*{PCA/SVD}\label{pcasvd}

\subsection*{Features}

In this part we have get such features, as \textbf{mean}, \textbf{count}, \textbf{vert\_symmetry}.
We choose this features, because  \textbf{mean}, \textbf{count} high correlated and interesting, how good pca and svd decorelate this features.
        
    \begin{tcolorbox}[breakable, size=fbox, boxrule=1pt, pad at break*=1mm,colback=cellbackground, colframe=cellborder]
\prompt{In}{incolor}{69}{\boxspacing}
\begin{Verbatim}[commandchars=\\\{\}]
\PY{n}{features} \PY{o}{=} \PY{p}{[}\PY{l+s+s1}{\PYZsq{}}\PY{l+s+s1}{mean}\PY{l+s+s1}{\PYZsq{}}\PY{p}{,} \PY{l+s+s1}{\PYZsq{}}\PY{l+s+s1}{count}\PY{l+s+s1}{\PYZsq{}}\PY{p}{,} \PY{l+s+s1}{\PYZsq{}}\PY{l+s+s1}{vert\PYZus{}symmetry}\PY{l+s+s1}{\PYZsq{}}\PY{p}{]}
\end{Verbatim}
\end{tcolorbox}

    \begin{tcolorbox}[breakable, size=fbox, boxrule=1pt, pad at break*=1mm,colback=cellbackground, colframe=cellborder]
\prompt{In}{incolor}{70}{\boxspacing}
\begin{Verbatim}[commandchars=\\\{\}]
\PY{n}{X\PYZus{}train} \PY{o}{=} \PY{n}{np}\PY{o}{.}\PY{n}{array}\PY{p}{(}\PY{p}{[}\PY{n}{train}\PY{p}{[}\PY{n}{feature}\PY{p}{]}\PY{o}{.}\PY{n}{values} \PY{k}{for} \PY{n}{feature} \PY{o+ow}{in} \PY{n}{features}\PY{p}{]}\PY{p}{)}\PY{o}{.}\PY{n}{T}
\PY{n}{X\PYZus{}test} \PY{o}{=} \PY{n}{np}\PY{o}{.}\PY{n}{array}\PY{p}{(}\PY{p}{[}\PY{n}{test}\PY{p}{[}\PY{n}{feature}\PY{p}{]}\PY{o}{.}\PY{n}{values} \PY{k}{for} \PY{n}{feature} \PY{o+ow}{in} \PY{n}{features}\PY{p}{]}\PY{p}{)}\PY{o}{.}\PY{n}{T}
\end{Verbatim}
\end{tcolorbox}

\subsection*{Standardize and SVD}

    \begin{tcolorbox}[breakable, size=fbox, boxrule=1pt, pad at break*=1mm,colback=cellbackground, colframe=cellborder]
\prompt{In}{incolor}{71}{\boxspacing}
\begin{Verbatim}[commandchars=\\\{\}]
\PY{n}{means} \PY{o}{=} \PY{n}{np}\PY{o}{.}\PY{n}{mean}\PY{p}{(}\PY{n}{X\PYZus{}train}\PY{p}{,} \PY{n}{axis}\PY{o}{=}\PY{l+m+mi}{0}\PY{p}{,} \PY{n}{keepdims}\PY{o}{=}\PY{k+kc}{True}\PY{p}{)}
\PY{n}{stds} \PY{o}{=} \PY{n}{np}\PY{o}{.}\PY{n}{std}\PY{p}{(}\PY{n}{X\PYZus{}train}\PY{p}{,} \PY{n}{axis}\PY{o}{=}\PY{l+m+mi}{0}\PY{p}{,} \PY{n}{keepdims}\PY{o}{=}\PY{k+kc}{True}\PY{p}{)}
\PY{n}{X\PYZus{}train\PYZus{}norm} \PY{o}{=} \PY{p}{(}\PY{n}{X\PYZus{}train} \PY{o}{\PYZhy{}} \PY{n}{means}\PY{p}{)} \PY{o}{/} \PY{n}{stds}

\PY{n+nb}{print}\PY{p}{(}\PY{l+s+s1}{\PYZsq{}}\PY{l+s+s1}{data scatter:}\PY{l+s+s1}{\PYZsq{}}\PY{p}{,} \PY{n}{np}\PY{o}{.}\PY{n}{sum}\PY{p}{(}\PY{n}{X\PYZus{}train} \PY{o}{*}\PY{o}{*} \PY{l+m+mi}{2}\PY{p}{)}\PY{p}{)}
\PY{n+nb}{print}\PY{p}{(}\PY{l+s+s1}{\PYZsq{}}\PY{l+s+s1}{data scatter after centering: }\PY{l+s+s1}{\PYZsq{}}\PY{p}{,} \PY{n}{np}\PY{o}{.}\PY{n}{sum}\PY{p}{(}\PY{p}{(}\PY{n}{X\PYZus{}train} \PY{o}{\PYZhy{}} \PY{n}{means}\PY{p}{)} \PY{o}{*}\PY{o}{*} \PY{l+m+mi}{2}\PY{p}{)}\PY{p}{)}
\PY{n+nb}{print}\PY{p}{(}\PY{l+s+s1}{\PYZsq{}}\PY{l+s+s1}{data scatter after standardize: }\PY{l+s+s1}{\PYZsq{}}\PY{p}{,} \PY{n}{np}\PY{o}{.}\PY{n}{sum}\PY{p}{(}\PY{n}{X\PYZus{}train\PYZus{}norm} \PY{o}{*}\PY{o}{*} \PY{l+m+mi}{2}\PY{p}{)}\PY{p}{)}
\end{Verbatim}
\end{tcolorbox}

    \begin{Verbatim}[commandchars=\\\{\}]
data scatter: 25337027.391464494
data scatter after centering:  1998165.1463920956
data scatter after standardize:  2999.9999999999995
    \end{Verbatim}

    \begin{tcolorbox}[breakable, size=fbox, boxrule=1pt, pad at break*=1mm,colback=cellbackground, colframe=cellborder]
\prompt{In}{incolor}{74}{\boxspacing}
\begin{Verbatim}[commandchars=\\\{\}]
\PY{n}{pca} \PY{o}{=} \PY{n}{PCA}\PY{p}{(}\PY{p}{)}
\PY{n}{transformed\PYZus{}array} \PY{o}{=} \PY{n}{pca}\PY{o}{.}\PY{n}{fit\PYZus{}transform}\PY{p}{(}\PY{n}{X\PYZus{}train\PYZus{}norm}\PY{p}{)}
\PY{n}{transformed} \PY{o}{=} \PY{n}{pd}\PY{o}{.}\PY{n}{DataFrame}\PY{p}{(}\PY{n}{transformed\PYZus{}array}\PY{p}{,} \PY{n}{columns}\PY{o}{=}\PY{p}{[}\PY{l+s+s1}{\PYZsq{}}\PY{l+s+s1}{PC}\PY{l+s+s1}{\PYZsq{}} \PY{o}{+} \PY{n+nb}{str}\PY{p}{(}\PY{n}{i}\PY{p}{)} \PY{k}{for} \PY{n}{i} \PY{o+ow}{in} \PY{n+nb}{range}\PY{p}{(}\PY{l+m+mi}{1}\PY{p}{,} \PY{n+nb}{len}\PY{p}{(}\PY{n}{features}\PY{p}{)} \PY{o}{+} \PY{l+m+mi}{1}\PY{p}{)}\PY{p}{]}\PY{p}{)}
\end{Verbatim}
\end{tcolorbox}

    \begin{tcolorbox}[breakable, size=fbox, boxrule=1pt, pad at break*=1mm,colback=cellbackground, colframe=cellborder]
\prompt{In}{incolor}{75}{\boxspacing}
\begin{Verbatim}[commandchars=\\\{\}]
\PY{n}{transformed}\PY{o}{.}\PY{n}{head}\PY{p}{(}\PY{p}{)}
\end{Verbatim}
\end{tcolorbox}

            \begin{tcolorbox}[breakable, size=fbox, boxrule=.5pt, pad at break*=1mm, opacityfill=0]
\prompt{Out}{outcolor}{75}{\boxspacing}
\begin{Verbatim}[commandchars=\\\{\}]
        PC1       PC2       PC3
0 -1.317348  1.511233  0.042066
1  0.479915  0.407563  0.375165
2  0.757433  0.808916  0.052019
3  0.324062  0.374046  0.216130
4  1.857682  0.182502  0.009792
\end{Verbatim}
\end{tcolorbox}
        
    \begin{tcolorbox}[breakable, size=fbox, boxrule=1pt, pad at break*=1mm,colback=cellbackground, colframe=cellborder]
\prompt{In}{incolor}{76}{\boxspacing}
\begin{Verbatim}[commandchars=\\\{\}]
\PY{n+nb}{print}\PY{p}{(}\PY{l+s+s1}{\PYZsq{}}\PY{l+s+s1}{pca components: }\PY{l+s+s1}{\PYZsq{}}\PY{p}{,} \PY{n}{pca}\PY{o}{.}\PY{n}{components\PYZus{}}\PY{p}{)}
\end{Verbatim}
\end{tcolorbox}

    \begin{Verbatim}[commandchars=\\\{\}]
pca components:  [[ 0.68032123  0.68058897 -0.27195897]
 [ 0.19338644  0.19122157  0.96230764]
 [ 0.70694038 -0.7072715  -0.00152458]]
    \end{Verbatim}

    \begin{tcolorbox}[breakable, size=fbox, boxrule=1pt, pad at break*=1mm,colback=cellbackground, colframe=cellborder]
\prompt{In}{incolor}{77}{\boxspacing}
\begin{Verbatim}[commandchars=\\\{\}]
\PY{n}{scatter} \PY{o}{=} \PY{n}{np}\PY{o}{.}\PY{n}{sum}\PY{p}{(}\PY{n}{X\PYZus{}train\PYZus{}norm} \PY{o}{*}\PY{o}{*} \PY{l+m+mi}{2}\PY{p}{)}
\PY{k}{for} \PY{n}{col\PYZus{}name} \PY{o+ow}{in} \PY{n}{transformed}\PY{p}{:}
    \PY{n}{col\PYZus{}scatter} \PY{o}{=} \PY{n}{np}\PY{o}{.}\PY{n}{sum}\PY{p}{(}\PY{n}{transformed}\PY{p}{[}\PY{n}{col\PYZus{}name}\PY{p}{]} \PY{o}{*}\PY{o}{*} \PY{l+m+mi}{2}\PY{p}{)}
    \PY{n}{scatter\PYZus{}percent} \PY{o}{=} \PY{l+m+mi}{100} \PY{o}{*} \PY{n}{col\PYZus{}scatter} \PY{o}{/} \PY{n}{scatter}
    \PY{n+nb}{print}\PY{p}{(}\PY{n}{col\PYZus{}name}\PY{p}{,} \PY{l+s+s1}{\PYZsq{}}\PY{l+s+s1}{contributes }\PY{l+s+si}{\PYZob{}0:.3f\PYZcb{}}\PY{l+s+s1}{, or }\PY{l+s+si}{\PYZob{}1:.2f\PYZcb{}}\PY{l+s+s1}{\PYZpc{}}\PY{l+s+s1}{, to the data scatter}\PY{l+s+s1}{\PYZsq{}}\PY{o}{.}\PY{n}{format}\PY{p}{(}\PY{n}{col\PYZus{}scatter}\PY{p}{,} \PY{n}{scatter\PYZus{}percent}\PY{p}{)}\PY{p}{)}
\end{Verbatim}
\end{tcolorbox}

    \begin{Verbatim}[commandchars=\\\{\}]
PC1 contributes 2046.691, or 68.22\%, to the data scatter
PC2 contributes 916.404, or 30.55\%, to the data scatter
PC3 contributes 36.905, or 1.23\%, to the data scatter
    \end{Verbatim}

\subsection*{Hidden ranking factor}

To obtain a hidden factor expressed in the 0-100 rank scale, we first rescale all the features to this range and then decompose the result using SVD and output the first component.

    \begin{tcolorbox}[breakable, size=fbox, boxrule=1pt, pad at break*=1mm,colback=cellbackground, colframe=cellborder]
\prompt{In}{incolor}{78}{\boxspacing}
\begin{Verbatim}[commandchars=\\\{\}]
\PY{k}{def} \PY{n+nf}{rescale}\PY{p}{(}\PY{n}{df}\PY{p}{)}\PY{p}{:}
    \PY{k}{return} \PY{p}{(}\PY{n}{df} \PY{o}{\PYZhy{}} \PY{n}{df}\PY{o}{.}\PY{n}{min}\PY{p}{(}\PY{p}{)}\PY{p}{)} \PY{o}{/} \PY{p}{(}\PY{n}{df}\PY{o}{.}\PY{n}{max}\PY{p}{(}\PY{p}{)} \PY{o}{\PYZhy{}} \PY{n}{df}\PY{o}{.}\PY{n}{min}\PY{p}{(}\PY{p}{)}\PY{p}{)} \PY{o}{*} \PY{l+m+mi}{100}

\PY{n}{rescaled\PYZus{}X} \PY{o}{=} \PY{n}{rescale}\PY{p}{(}\PY{n}{X\PYZus{}train}\PY{p}{)}
\PY{n}{U}\PY{p}{,} \PY{n}{s}\PY{p}{,} \PY{n}{V} \PY{o}{=} \PY{n}{np}\PY{o}{.}\PY{n}{linalg}\PY{o}{.}\PY{n}{svd}\PY{p}{(}\PY{n}{rescaled\PYZus{}X}\PY{p}{)}
\PY{n}{contribution} \PY{o}{=} \PY{l+m+mi}{100} \PY{o}{*} \PY{n}{s}\PY{p}{[}\PY{l+m+mi}{0}\PY{p}{]} \PY{o}{*}\PY{o}{*} \PY{l+m+mi}{2} \PY{o}{/} \PY{n}{np}\PY{o}{.}\PY{n}{sum}\PY{p}{(}\PY{n}{rescaled\PYZus{}X} \PY{o}{*}\PY{o}{*} \PY{l+m+mi}{2}\PY{p}{)}
\end{Verbatim}
\end{tcolorbox}

    \begin{tcolorbox}[breakable, size=fbox, boxrule=1pt, pad at break*=1mm,colback=cellbackground, colframe=cellborder]
\prompt{In}{incolor}{79}{\boxspacing}
\begin{Verbatim}[commandchars=\\\{\}]
\PY{n+nb}{print}\PY{p}{(}\PY{l+s+s1}{\PYZsq{}}\PY{l+s+s1}{First component:}\PY{l+s+s1}{\PYZsq{}}\PY{p}{,} \PY{n}{V}\PY{p}{[}\PY{l+m+mi}{0}\PY{p}{]}\PY{p}{)}
\PY{n+nb}{print}\PY{p}{(}\PY{l+s+s1}{\PYZsq{}}\PY{l+s+s1}{Its contribution to the data scatter: }\PY{l+s+si}{\PYZob{}0:.3f\PYZcb{}}\PY{l+s+s1}{\PYZpc{}}\PY{l+s+s1}{\PYZsq{}}\PY{o}{.}\PY{n}{format}\PY{p}{(}\PY{n}{contribution}\PY{p}{)}\PY{p}{)}
\end{Verbatim}
\end{tcolorbox}

    \begin{Verbatim}[commandchars=\\\{\}]
First component: [-0.33506822 -0.93161987 -0.14076113]
Its contribution to the data scatter: 99.638\%
    \end{Verbatim}
    
   That the contribution of the first component is much higher than in the previous case should not be surprising: the data is not centered, so its mean is the source of most of the data scatter.

\subsection*{Visualization}

At first visualize all labels.

    \begin{tcolorbox}[breakable, size=fbox, boxrule=1pt, pad at break*=1mm,colback=cellbackground, colframe=cellborder]
\prompt{In}{incolor}{99}{\boxspacing}
\begin{Verbatim}[commandchars=\\\{\}]
\PY{k}{def} \PY{n+nf}{pairplot}\PY{p}{(}\PY{n}{df}\PY{p}{,} \PY{n}{title}\PY{p}{)}\PY{p}{:}
    \PY{n}{plot} \PY{o}{=} \PY{n}{sns}\PY{o}{.}\PY{n}{pairplot}\PY{p}{(}\PY{n}{df}\PY{p}{,} \PY{n+nb}{vars}\PY{o}{=}\PY{n}{df}\PY{o}{.}\PY{n}{columns}\PY{p}{[}\PY{p}{:}\PY{o}{\PYZhy{}}\PY{l+m+mi}{1}\PY{p}{]}\PY{p}{,} \PY{n}{hue}\PY{o}{=}\PY{l+s+s1}{\PYZsq{}}\PY{l+s+s1}{label}\PY{l+s+s1}{\PYZsq{}}\PY{p}{)}
    \PY{n}{plot}\PY{o}{.}\PY{n}{fig}\PY{o}{.}\PY{n}{suptitle}\PY{p}{(}\PY{n}{title}\PY{p}{)}

\PY{n}{df\PYZus{}standart} \PY{o}{=} \PY{n}{pd}\PY{o}{.}\PY{n}{DataFrame}\PY{p}{(}\PY{n}{X\PYZus{}train\PYZus{}norm}\PY{p}{,} \PY{n}{columns}\PY{o}{=}\PY{n}{features}\PY{p}{)} 
\PY{n}{df\PYZus{}rescale} \PY{o}{=} \PY{n}{pd}\PY{o}{.}\PY{n}{DataFrame}\PY{p}{(}\PY{n}{rescaled\PYZus{}X}\PY{p}{,} \PY{n}{columns}\PY{o}{=}\PY{n}{features}\PY{p}{)}
\PY{n}{df\PYZus{}transform} \PY{o}{=} \PY{n}{pd}\PY{o}{.}\PY{n}{DataFrame}\PY{p}{(}\PY{n}{transformed\PYZus{}array}\PY{p}{,} \PY{n}{columns}\PY{o}{=}\PY{p}{[}\PY{l+s+s1}{\PYZsq{}}\PY{l+s+s1}{PC}\PY{l+s+s1}{\PYZsq{}} \PY{o}{+} \PY{n+nb}{str}\PY{p}{(}\PY{n}{i}\PY{p}{)} \PY{k}{for} \PY{n}{i} \PY{o+ow}{in} \PY{n+nb}{range}\PY{p}{(}\PY{l+m+mi}{1}\PY{p}{,} \PY{n+nb}{len}\PY{p}{(}\PY{n}{features}\PY{p}{)} \PY{o}{+} \PY{l+m+mi}{1}\PY{p}{)}\PY{p}{]}\PY{p}{)}

\PY{k}{for} \PY{n}{df} \PY{o+ow}{in} \PY{p}{(}\PY{n}{df\PYZus{}standart}\PY{p}{,} \PY{n}{df\PYZus{}rescale}\PY{p}{,} \PY{n}{df\PYZus{}transform}\PY{p}{)}\PY{p}{:}
    \PY{n}{df}\PY{p}{[}\PY{l+s+s1}{\PYZsq{}}\PY{l+s+s1}{label}\PY{l+s+s1}{\PYZsq{}}\PY{p}{]} \PY{o}{=} \PY{n}{train}\PY{p}{[}\PY{l+s+s1}{\PYZsq{}}\PY{l+s+s1}{label}\PY{l+s+s1}{\PYZsq{}}\PY{p}{]}

\PY{n}{pairplot}\PY{p}{(}\PY{n}{df\PYZus{}standart}\PY{p}{,} \PY{l+s+s1}{\PYZsq{}}\PY{l+s+s1}{Standardized (mean=0, std=1)}\PY{l+s+s1}{\PYZsq{}}\PY{p}{)}
\PY{n}{pairplot}\PY{p}{(}\PY{n}{df\PYZus{}rescale}\PY{p}{,} \PY{l+s+s1}{\PYZsq{}}\PY{l+s+s1}{Normalized (range=[0, 100])}\PY{l+s+s1}{\PYZsq{}}\PY{p}{)}
\PY{n}{pairplot}\PY{p}{(}\PY{n}{df\PYZus{}transform}\PY{p}{,} \PY{l+s+s1}{\PYZsq{}}\PY{l+s+s1}{Primary components}\PY{l+s+s1}{\PYZsq{}}\PY{p}{)}
\end{Verbatim}
\end{tcolorbox}

    \begin{center}
    \adjustimage{max size={0.9\linewidth}{0.9\paperheight}}{PCA-SVD/output_17_0.png}
    \end{center}
    { \hspace*{\fill} \\}
    
    \begin{center}
    \adjustimage{max size={0.9\linewidth}{0.9\paperheight}}{PCA-SVD/output_17_1.png}
    \end{center}
    { \hspace*{\fill} \\}
    
    \begin{center}
    \adjustimage{max size={0.9\linewidth}{0.9\paperheight}}{PCA-SVD/output_17_2.png}
    \end{center}
    { \hspace*{\fill} \\}
    
 Too difficult understand something in case with many labels. Visualize only first 4 digits.
    
    \begin{tcolorbox}[breakable, size=fbox, boxrule=1pt, pad at break*=1mm,colback=cellbackground, colframe=cellborder]
\prompt{In}{incolor}{100}{\boxspacing}
\begin{Verbatim}[commandchars=\\\{\}]
\PY{n}{indexes} \PY{o}{=} \PY{n}{train}\PY{p}{[}\PY{l+s+s1}{\PYZsq{}}\PY{l+s+s1}{label}\PY{l+s+s1}{\PYZsq{}}\PY{p}{]}\PY{o}{.}\PY{n}{values} \PY{o}{\PYZlt{}}\PY{o}{=} \PY{l+m+mi}{3}
\PY{n}{df\PYZus{}standart}\PY{p}{,} \PY{n}{df\PYZus{}rescale}\PY{p}{,} \PY{n}{df\PYZus{}transform} \PY{o}{=} \PY{p}{[}\PY{n}{df}\PY{p}{[}\PY{n}{indexes}\PY{p}{]} \PY{k}{for} \PY{n}{df} \PY{o+ow}{in} \PY{p}{(}\PY{n}{df\PYZus{}standart}\PY{p}{,} \PY{n}{df\PYZus{}rescale}\PY{p}{,} \PY{n}{df\PYZus{}transform}\PY{p}{)}\PY{p}{]}

\PY{n}{pairplot}\PY{p}{(}\PY{n}{df\PYZus{}standart}\PY{p}{,} \PY{l+s+s1}{\PYZsq{}}\PY{l+s+s1}{Standardized (mean=0, std=1)}\PY{l+s+s1}{\PYZsq{}}\PY{p}{)}
\PY{n}{pairplot}\PY{p}{(}\PY{n}{df\PYZus{}rescale}\PY{p}{,} \PY{l+s+s1}{\PYZsq{}}\PY{l+s+s1}{Normalized (range=[0, 100])}\PY{l+s+s1}{\PYZsq{}}\PY{p}{)}
\PY{n}{pairplot}\PY{p}{(}\PY{n}{df\PYZus{}transform}\PY{p}{,} \PY{l+s+s1}{\PYZsq{}}\PY{l+s+s1}{Primary components}\PY{l+s+s1}{\PYZsq{}}\PY{p}{)}
\end{Verbatim}
\end{tcolorbox}

    \begin{center}
    \adjustimage{max size={0.9\linewidth}{0.9\paperheight}}{PCA-SVD/output_18_0.png}
    \end{center}
    { \hspace*{\fill} \\}
    
    \begin{center}
    \adjustimage{max size={0.9\linewidth}{0.9\paperheight}}{PCA-SVD/output_18_1.png}
    \end{center}
    { \hspace*{\fill} \\}
    
    \begin{center}
    \adjustimage{max size={0.9\linewidth}{0.9\paperheight}}{PCA-SVD/output_18_2.png}
    \end{center}
    { \hspace*{\fill} \\}
    
    Obviously, since we only use line plots and pairwise scatter plots, the difference between standardization and normalization amounts to relabeling of the axes (which might be helpful by itself because it makes interpreting the coordinates in the graph conceptually easier).
    
    However, if we were, for example, using a 3D scatter plot with fixed scales of the axes, normalization into the [0, 1] range would help a lot as it guarantees that the features have the same scale (if there are no outliers).
    
    PCA doesn’t seem to help a lot with distinguishing the points with digits labeling. All features can divede ones from others class, but with other digits all features have problems. But PC2 can divide all digits on three gropus: \{2, 4, 6\}, \{1\}, \{0, 3, 5, 7, 8, 9\}
